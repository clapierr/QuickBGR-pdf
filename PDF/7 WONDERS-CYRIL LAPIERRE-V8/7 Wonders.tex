\documentclass{scrartcl}%
\usepackage[T1]{fontenc}%
\usepackage[utf8]{inputenc}%
\usepackage{lmodern}%
\usepackage{textcomp}%
\usepackage{lastpage}%
\usepackage{geometry}%
\geometry{margin=0.7in}%
\usepackage{fancyhdr}%
%
\usepackage{sectsty}%
\usepackage{graphicx}%
\title{\includegraphics[width=8cm,height=4cm,keepaspectratio]{C:/Users/badcy/Documents/Unif/BAC 3/Q1/Projet individuel/2223_INFOB318_QuickBGR/code/my_app/static/img/tmp/Box-7 Wonders.jpg}\break 7 Wonders }%
\author{Cyril Lapierre}%
\date{\today \break Tags: Ancient, Card Game, City Building, Civilization, Economic}%
\fancypagestyle{header}{%
\renewcommand{\headrulewidth}{1pt}%
\renewcommand{\footrulewidth}{1pt}%
\fancyhead{%
}%
\fancyfoot{%
}%
\fancyhead[L]{%
Page \thepage%
}%
\fancyhead[C]{%
\includegraphics[width=4cm,height=1cm,keepaspectratio]{C:/Users/badcy/Documents/Unif/BAC 3/Q1/Projet individuel/2223_INFOB318_QuickBGR/code/my_app/static/img/logo/QuickBGR-NB.png}%
}%
\fancyhead[R]{%
\today%
}%
\fancyfoot[C]{%
\includegraphics[width=4cm,height=1cm,keepaspectratio]{C:/Users/badcy/Documents/Unif/BAC 3/Q1/Projet individuel/2223_INFOB318_QuickBGR/code/my_app/static/img/logo/QuickBGR-NB.png}%
}%
}%
%
\begin{document}%
\normalsize%
\maketitle\thispagestyle{header}%
\pagestyle{header}%
\section{ Intro
}%
\label{sec:Intro}%
\rule{18cm}{0.07cm}\break%
Vous avez 3 âges pour développer une grande cité du monde antique et construire une des 7 merveilles du
%
monde. à l’aide de vos cartes, développez le côté militaire ou scientifique de votre ville. Mettez en valeur
%
votre cité en construisant des bâtiments prestigieux, sans oublier de développer votre économie.
%
Gérez au mieux votre cité et laissez une trace dans l'histoire...


%
\section{ Mise en place
}%
\label{sec:Miseenplace}%
\rule{18cm}{0.07cm}\break%
%
\begin{center}\includegraphics[width=6cm,height=6cm,keepaspectratio]{C:/Users/badcy/Documents/Unif/BAC 3/Q1/Projet individuel/2223_INFOB318_QuickBGR/code/my_app/static/img/tmp/1.jpg}\end{center}%

%

%
\begin{enumerate}%
\item%
%
 Enlevez les %
\textbf{cartes non{-}utilisées}%
, selon le nombre de joueurs (chiffre en bas de la carte).
%
\item%
%
 %
\textbf{Guilde}%
\textit{ }%
 : Dans le paquet de l'%
\textbf{Âge 3}%
\textit{ }%
 : ajoutez le nombre de Guildes (8 cartes face recto violet) selon le nombre de joueurs (nombre de joueurs + 2).
%
\item%
%
 Faites un tas avec les 3 sortes de %
\textbf{jetons Militaires}%
\textit{ }%
 (rouges).
%
\end{enumerate}%
%
\begin{center}\includegraphics[width=4cm,height=4cm,keepaspectratio]{C:/Users/badcy/Documents/Unif/BAC 3/Q1/Projet individuel/2223_INFOB318_QuickBGR/code/my_app/static/img/tmp/2.jpg}\end{center}%

%

%
Chaque joueur reçoit :
%
\begin{itemize}%
\item%
%
 1 carte Merveille
%
\item%
%
 3 pièces d’or %
\includegraphics[width=1cm,height=1cm,keepaspectratio]{C:/Users/badcy/Documents/Unif/BAC 3/Q1/Projet individuel/2223_INFOB318_QuickBGR/code/my_app/static/img/tmp/3.jpg}%

%

%
\item%
%
 7 cartes de l'Âge 1 
%
\end{itemize}

%
\section{ Tour de jeu
}%
\label{sec:Tourdejeu}%
\rule{18cm}{0.07cm}\break%
%
\begin{center}\includegraphics[width=2cm,height=2cm,keepaspectratio]{C:/Users/badcy/Documents/Unif/BAC 3/Q1/Projet individuel/2223_INFOB318_QuickBGR/code/my_app/static/img/tmp/4.jpg}\end{center}%

%

%
Une partie se joue en 3 âges représentés par 3 piles de cartes numérotés I, II et III. Pour chaque âge, procédez comme suit :


%
\subsection{ CHOISIR UNE CARTE
}%
\label{subsec:CHOISIRUNECARTE}%
\begin{description}%
\item[{-} ]%
%
 Choisissez 1 %
\textbf{carte}%
\textit{ }%
 parmi les cartes que vous avez en main.
%
\end{description}

%
\subsection{ UTILISATION DE LA CARTE CHOISIE :
}%
\label{subsec:UTILISATIONDELACARTECHOISIE}%
Il y a 3 actions possibles avec cette carte :
%
\begin{description}%
\item[{-} ]%
%
 %
\textbf{CONSTRUIRE LE BÂTIMENT}%
\textit{ }%
 :
%
\end{description}%
\begin{itemize}%
\item%
%
 Le coût de construction est indiqué par un nombre de ressources à dépenser (à gauche de la carte). Si la zone est %
\textbf{vide}%
, la carte est gratuite.
%
\item%
%
 Vous devez posséder les ressources nécessaires dans votre Cité. Les ressources sont générées grâce aux cartes Production (marron et grises) que vous avez précédemment placées.
%
\item%
%
 Si vous possèdez déjà un Bâtiment avec le même %
\textbf{symbole}%
\textit{ }%
 que celui illustré à côté du coût , vous ne payez rien (Chaînage).
%
\item%
%
 S'il vous manque des ressources, vous pouvez les acheter à vos voisins (de droite et de gauche) en faisant du %
\textit{Commerce}%
. Dans ce cas, donnez 2 pièces à ce voisins pour utiliser la(les)ressource(s) produite(s) par l'une de ses cartes marron ou grise.
%
\item%
%
 On ne peut pas construire 2 Bâtiments identiques (même nom).
%
\item%
%
 Posez la carte au{-}dessus de votre Merveille.
%
\end{itemize}%
\begin{description}%
\item[{-} ]%
%
 %
\textbf{CONSTRUIRE UNE ÉTAPE DE SA MERVEILLE}%
\textit{ }%
 :
%
\end{description}%
\begin{itemize}%
\item%
%
 Si vous possédez les ressources nécessaires : coût indiqué sur chaque emplacement des Merveilles.
%
\item%
%
 Possibilité de faire du Commerce avec vos voisins de droite et de gauche pour obtenir des ressources manquantes. Glissez la carte côté %
\textit{verso}%
\textit{ }%
 visible sous l’emplacement %
\textbf{libre}%
\textit{ }%
 le plus à gauche de votre Merveille.
%
\end{itemize}%
\begin{description}%
\item[{-} ]%
%
 %
\textbf{DÉFAUSSER LA CARTE POUR OBTENIR 3 PIÈCES D’OR.}%

%
\end{description}

%
\subsection{ PASSEZ LES CARTES RESTANTES
}%
\label{subsec:PASSEZLESCARTESRESTANTES}%
\begin{description}%
\item[{-} ]%
%
 Passez les cartes restantes de votre main à votre voisin : de gauche (1er et 3ème Âge)/de droite (2ème Âge).
%
\item[{-} ]%
%
 Au sixième tour, choisissez une carte parmi les 2 cartes que vous venez de recevoir et défaussez l'autre. Vous n'obtenez pas de pièces d'or pour la carte défaussée ainsi.
%
\end{description}

%
\subsection{ FIN D’UN ÂGE
}%
\label{subsec:FINDUNGE}%
\begin{description}%
\item[{-} ]%
%
 Résolvez les conflits Militaires : Comparez le nombre de %
\textbf{Boucliers}%
\textit{ }%
 dans votre Cité avec celui de vos voisins de droite et de gauche.
%
\item[{-} ]%
%
 Si vous disposez d’un total supérieur à celui d’une Cité voisine, gagnez 1 jeton Militaire (Âge 1 = 1 point / Âge 2 = 3 points /Âge 3 = 5 points) ou 1 jeton malus (%
\textit{{-}1 point}%
) si votre total est inférieur. En cas d'égalité, aucun jeton n’est pris.
%
\end{description}%
\textbf{DESCRIPTION DES CARTES}%

%
\begin{description}%
\item[{-} ]%
%
 %
\textbf{Cartes marron}%
\textit{ }%
 : Matières premières. Elles produisent à chaque tour les ressources indiquées sur la carte.
%
\item[{-} ]%
%
 %
\textbf{Cartes grises}%
\textit{ }%
 : Produits manufacturés. Elles produisent des ressources plus rares.
%
\item[{-} ]%
%
 %
\textbf{Cartes bleues}%
\textit{ }%
 : Bâtiments civils : Elles rapportent des points de victoire en fin de partie.
%
\item[{-} ]%
%
 %
\textbf{Cartes rouges}%
\textit{ }%
 : Bâtiments Militaires : Augmentent la force Militaire (Boucliers). Permettent de gagner militairement sur ses voisins et de remporter des points de victoire lors des conflits de fin d’Âge.
%
\item[{-} ]%
%
 %
\textbf{Cartes jaunes}%
\textit{ }%
 : Bâtiments commerciaux : Produisent de l’argent et des avantages commerciaux. (Commerce plus avantageux avec les voisins ou production complémentaire de ressources).
%
\item[{-} ]%
%
 %
\textbf{Cartes violette}%
\textit{ }%
 : Guildes : Permettent de gagner des points spéciaux en fin de partie.
%
\item[{-} ]%
%
 %
\textbf{Cartes vertes}%
\textit{ }%
 : Bâtiments scientifiques : Permettent de marquer des P.V selon le nombre de symboles. Les 2 façons de scorer sont cumulables :
%
\end{description}%
\begin{itemize}%
\item%
%
 Symboles Identiques = total multiplié au carré.
%
\item%
%
 3 symboles différents = 7 points
%
\end{itemize}

%
\section{ Fin de partie
}%
\label{sec:Findepartie}%
\rule{18cm}{0.07cm}\break%
La partie se termine à la fin du 3ème Âge. Comptez vos points de victoire. Le joueur en possédant le plus, remporte la partie :
%
\begin{enumerate}%
\item%
%
 Additionnez les points de victoire Militaires.
%
\item%
%
 Chaque lot de 3 pièces = 1 point de victoire.
%
\item%
%
 Points de victoire des différentes combinaisons des Bâtiments %
\textbf{\textit{scientifiques}}%
.
%
\end{enumerate}%
Rajoutez les points de victoires indiqués sur :
%
\begin{enumerate}%
\item%
%
 %
\textit{Les cartes Merveilles}%
.
%
\item%
%
 %
\textit{Les Bâtiments civils}%
. 
%
\item%
%
 %
\textit{Les cartes spéciales jaunes}%
.
%
\item%
%
 %
\textit{Les cartes Guildes}%
.%
\end{enumerate}

%
\end{document}
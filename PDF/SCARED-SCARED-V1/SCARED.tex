\documentclass{article}%
\usepackage[T1]{fontenc}%
\usepackage[utf8]{inputenc}%
\usepackage{lmodern}%
\usepackage{textcomp}%
%
\title{SCARED}%
\author{SCARED}%
\date{\today}%
%
\begin{document}%
\pagestyle{empty}%
\normalsize%
\maketitle%
\section{ Set up
}%
\label{sec:Setup}%
{-} Préparez l'équivalent de %
\textbf{4/5 Missions}%
*4/5 Missions%
\textbf{.Chaque Mission est composée d'une carte }%
*.Chaque Mission est composée d'une carte %
\textbf{Codes secrets}%
*Codes secrets** et d un set de cartes %
\textbf{Disciple}%
*Disciple%
\textbf{,}%
{-} Choisissez une Mission et distribuez une carte à chaque joueur. Posez au %
\textbf{centre}%
*centre** de la table la carte %
\textbf{Codes secrets}%
*Codes secrets%
\textbf{.
}%
{-} Pour chaque mission, les mots des cartes %
\textbf{Disciples}%
*Disciples** ont la m me couleur que ceux de la carte Codes Secrets alors que les cartes Infiltr s ont une couleur%
\textbf{différente}%
*différente%
\textbf{.
}%
{-} %
\textbf{Le premier joueur}%
*Le premier joueur** choisit un chiffre entre 1 et 10 pour s lectionner le %
\textbf{Mot de passe}

%
\section{ Round of play
}%
\label{sec:Roundofplay}%
%
\textbf{Chacun leur tour, les joueurs annoncent un mot en rapport avec le mot de passe s lectionné. L'infiltré a un mot de passe légérement différent des autres et le journaliste, qui ne connait pas le mot de passe, doit improviser en s'inspirant des mots précédemment annoncés}%
*Chacun leur tour, les joueurs annoncent un mot en rapport avec le mot de passe s lectionné. L'infiltré a un mot de passe légérement différent des autres et le journaliste, qui ne connait pas le mot de passe, doit improviser en s'inspirant des mots précédemment annoncés%
\textbf{.
}

%
\subsection{ Donner un indice
}%
\label{subsec:Donnerunindice}%
{-} %
\textit{Les Disciples}%
{-} %
\textit{Les Infiltrés}%
Les Infiltrés* : Ils reçoivent un Mot de passe légérement différent de celui des %
\textit{Disciples}%
Disciples%
\textit{. Leur objectif est de se faire passer pour des Disciples.
}%
{-} %
\textit{Le Journaliste}

%
\subsection{ Débattre
}%
\label{subsec:Dbattre}%

%
\section{ End of game
}%
\label{sec:Endofgame}%

%
\end{document}
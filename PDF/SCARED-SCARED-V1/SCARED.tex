\documentclass{article}%
\usepackage[T1]{fontenc}%
\usepackage[utf8]{inputenc}%
\usepackage{lmodern}%
\usepackage{textcomp}%
%
\title{SCARED}%
\author{SCARED}%
\date{\today}%
%
\begin{document}%
\pagestyle{empty}%
\normalsize%
\maketitle%
\section{ Set up
}%
\label{sec:Setup}%
\begin{enumerate}%
\item%
%
\item%
 Préparez l'équivalent de **4/5 Missions**.Chaque Mission est composée d'une carte **Codes secrets** et d'un set de cartes **Disciple**,** Infiltrés** et Journaliste reparties selon le nombre de joueurs.
%
\end{enumerate}%
2. Choisissez une Mission et distribuez une carte à chaque joueur. Posez au %
\textbf{centre}%
\textit{ de la table la carte }%
\textbf{Codes secrets}%
\textit{.
}%
3. Pour chaque mission, les mots des cartes %
\textbf{Disciples}%
\textit{ ont la même couleur que ceux de la carte Codes Secrets alors que les cartes Infiltrés ont une couleur }%
\textbf{différente}%
\textit{.
}%
4. %
\textbf{Le premier joueur}%
\textit{ choisit un chiffre entre 1 et 10 pour s lectionner le }%
\textbf{Mot de passe}%
\textit{ sur sa carte.
}

%
\section{ Round of play
}%
\label{sec:Roundofplay}%
\textbf{Chacun leur tour, les joueurs annoncent un mot en rapport avec le mot de passe sélectionné. L'infiltré a un mot de passe légérement différent des autres et le journaliste, qui ne connait pas le mot de passe, doit improviser en s'inspirant des mots précédemment annoncés}%
\textit{.
}

%
\subsection{ Donner un indice
}%
\label{subsec:Donnerunindice}%
\begin{enumerate}%
\item%
%
\item%
 Chacun leur tour, les joueurs annonce un Indice (un mot) en rapport avec le Mot de passe préalablement sélectionné indiqué sur leur carte.
%
\end{enumerate}%
2. A part le Journaliste (qui connait son rôle), les autres joueurs ne savent pas s'ils sont Disciples ou Infiltrés.
%
3. %
\textit{Les Disciples}%
\textit{ : Ils reçoivent tous le même Mot de passe. Leur objectif est de démasquer les Infiltrés et le Journaliste.
}%
4. %
\textit{Les Infiltrés}%
\textit{ : Ils reçoivent un Mot de passe légérement différent de celui des }%
\textit{Disciples}%
\textit{. Leur objectif est de se faire passer pour des Disciples.
}%
5. %
\textit{Le Journaliste}%
\textit{ : Il ne reçoit aucun Mot de passe. Son but est de faire croire qu'il en a un, tout en essayant de deviner celui des Disciples.
}

%
\subsection{ Débattre
}%
\label{subsec:Dbattre}%
{-} Une fois que tous les joueurs ont donné leur Indice, débattez entre vous pour déterminer qui peut être le Journaliste ou un Infiltré.
%
{-} Votez tous en même temps en pointant de votre doigt la personne que vous soupçonnez.
%
{-} Le joueur désigné par la majorité des votants est éliminé :
%
\begin{enumerate}%
\item%
Le joueur désigné par la majorité des votants est éliminé :
%
\item%
 Si c’était le Journaliste, il peut encore tenter de deviner le Mot de passe des Disciples. Il l'annonce à haute voix puis vérifie sur la carte Codes secrets. S’il réussit la partie s’arrête et il marque 3 points.S’il échoue, la partie continue.
%
\item%
 Si ce n’était pas le Journaliste, il regarde discrètement la carte Codes secrets et annonce aux autres joueurs s’il était un Disciple (il a le même mot que sur la carte Codes secrets) ou un Infiltré (il a un mot différent).
%
\end{enumerate}%
{-} Enchaînez ainsi les tours jusqu’à ce que tous les Infiltrés et le Journaliste aient été éliminés ou qu’il ne reste plus que 2 joueurs :
%
\begin{enumerate}%
\item%
Enchaînez ainsi les tours jusqu’à ce que tous les Infiltrés et le Journaliste aient été éliminés ou qu’il ne reste plus que 2 joueurs :
%
\item%
 Les Disciples gagnent si tous les Infiltrés et le Journaliste ont été éliminés, ils marquent alors tous 1 point (y compris
%
\end{enumerate}%
\begin{enumerate}%
\item%
s Disciples déjà éliminés).
%
\item%
 Les Infiltrés gagnent si l'un d'eux ne s'est pas fait éliminer, ils marquent alors tous 2 points (y compris les Infiltrés déjà éliminés).
%
\item%
 Le Journaliste gagne s'il a deviné le mot de passe des Disciples ou s'il n'a pas été éliminé, il marque alors 3 points.
%
\end{enumerate}

%
\section{ End of game
}%
\label{sec:Endofgame}%
\textbf{Le premier joueur atteignant 5 points est déclaré vainqueur}

%
\end{document}
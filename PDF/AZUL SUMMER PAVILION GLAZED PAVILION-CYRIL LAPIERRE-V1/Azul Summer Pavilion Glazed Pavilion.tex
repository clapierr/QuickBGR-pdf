\documentclass{scrartcl}%
\usepackage[T1]{fontenc}%
\usepackage[utf8]{inputenc}%
\usepackage{lmodern}%
\usepackage{textcomp}%
\usepackage{lastpage}%
\usepackage{geometry}%
\geometry{margin=0.7in}%
\usepackage{xcolor}%
\usepackage{fancyhdr}%
%
\usepackage{sectsty}%
\usepackage{graphicx}%
\usepackage{xcolor}%
\definecolor{mygreen}{RGB}{58,170,53}%
\title{Azul Summer Pavilion Glazed Pavilion}%
\author{Cyril Lapierre}%
\date{\today}%
\fancypagestyle{header}{%
\renewcommand{\headrulewidth}{1pt}%
\renewcommand{\footrulewidth}{1pt}%
\fancyhead{%
}%
\fancyfoot{%
}%
\fancyhead[L]{%
Page \thepage%
}%
\fancyhead[C]{%
\includegraphics[width=4cm,height=1cm,keepaspectratio]{C:/Users/badcy/Documents/Unif/BAC 3/Q1/Projet individuel/2223_INFOB318_QuickBGR/code/my_app/static/img/logo/QuickBGR-black-green.png}%
}%
\fancyhead[R]{%
\today%
}%
\fancyfoot[C]{%
\includegraphics[width=4cm,height=1cm,keepaspectratio]{C:/Users/badcy/Documents/Unif/BAC 3/Q1/Projet individuel/2223_INFOB318_QuickBGR/code/my_app/static/img/logo/QuickBGR-black-green.png}%
}%
}%
%
\begin{document}%
\normalsize%
\maketitle\thispagestyle{header}%
\pagestyle{header}%
\sectionfont{\color{blue}}%
\subsectionfont{\color{blue}}%
\subsubsectionfont{\color{blue}}%
\section{ Mise en place
}%
\label{sec:Miseenplace}%
\textcolor{blue}{\rule{18cm}{0.07cm}}\break%
%
\includegraphics[width=4cm,height=4cm,keepaspectratio]{C:/Users/badcy/Documents/Unif/BAC 3/Q1/Projet individuel/2223_INFOB318_QuickBGR/code/my_app/static/img/tmp/5.jpg}%
%
\includegraphics[width=4cm,height=4cm,keepaspectratio]{C:/Users/badcy/Documents/Unif/BAC 3/Q1/Projet individuel/2223_INFOB318_QuickBGR/code/my_app/static/img/tmp/6.jpg}%
%
\includegraphics[width=4cm,height=4cm,keepaspectratio]{C:/Users/badcy/Documents/Unif/BAC 3/Q1/Projet individuel/2223_INFOB318_QuickBGR/code/my_app/static/img/tmp/7.jpg}%

%
\begin{description}%
\item[{-} ]%
%
 Placez la totalité %
\textcolor{blue}{%
\textbf{des tuiles}%
}%
\textit{ }%
 dans le sac prévu à cet effet.
%
\item[{-} ]%
%
 %
\textcolor{blue}{%
\textbf{Chaque joueur}%
}%
\textit{ }%
 reçoit un plateau personnel.
%
\item[{-} ]%
%
 Placez sur le plateau de la piste de score :
%
\end{description}%
\begin{itemize}%
\item%
%
 Un jeton marqueur de score sur la %
\textcolor{blue}{%
\textbf{case 5}%
}%
\textit{ }%
 de la piste de score.
%
\item%
%
 Le marqueur de manche sur le 1er emplacement sous l’illustration de tuile mauve.
%
\item%
%
 10 tuiles (piochées au hasard dans le sac) à disposer sur les %
\textcolor{blue}{%
\textbf{10 espaces}%
}%
\textit{ }%
 de la rosace centrale.
%
\end{itemize}%
\begin{description}%
\item[{-} ]%
%
 Disposez les cercles de fabrique au centre de la table selon le nombre de joueurs (5/7/9 cercles pour 2/3/4 joueurs).
%
\item[{-} ]%
%
 Remplissez chaque cercle de fabrique avec %
\textcolor{blue}{%
\textbf{4 tuiles piochées}%
}%
\textit{ }%
 aléatoirement dans le sac.
%
\item[{-} ]%
%
 Placez le jeton %
\textcolor{blue}{%
\textbf{1er joueur}%
}%
\textit{ }%
 au milieu des cercle de fabrique et la tour à %
\textcolor{blue}{%
\textbf{défausse}%
}%
\textit{ }%
 à proximité des joueurs.
%
\end{description}

%
\sectionfont{\color{mygreen}}%
\subsectionfont{\color{mygreen}}%
\subsubsectionfont{\color{mygreen}}%
\section{ Round of play
}%
\label{sec:Roundofplay}%
\textcolor{mygreen}{\rule{18cm}{0.07cm}}\break%
%
\begin{center}\includegraphics[width=4cm,height=4cm,keepaspectratio]{C:/Users/badcy/Documents/Unif/BAC 3/Q1/Projet individuel/2223_INFOB318_QuickBGR/code/my_app/static/img/tmp/8.jpg}\end{center}%

%
Une partie est composée de %
\textcolor{mygreen}{%
\textbf{6 manches}%
}%
. Le marqueur de manche indique la couleur joker de la manche en cours. Chaque manche est composée de 3 étapes.


%
\subsection{ RÉCUPÉRER DES TUILES :
}%
\label{subsec:RCUPRERDESTUILES}%
\begin{description}%
\item[{-} ]%
%
 Chacun leur %
\textcolor{mygreen}{%
\textbf{tour les joueurs}%
}%
\textit{ }%
 peuvent récupérer des tuiles soit : Sur un cercle de fabrique :
%
\end{description}%
\begin{itemize}%
\item%
%
 Choisissez une couleur autre que la couleur joker du tour et ramassez la totalité des tuiles de celle{-}ci sur le cercle.
%
\item%
%
 Si le cercle sélectionné contient une ou plusieurs tuiles de la %
\textcolor{mygreen}{%
\textbf{couleur joker}%
}%
, vous devez en prendre une (et une seule).
%
\item%
%
 Le reste de tuiles encore présentes sur le cercle sont défaussé au centre de la table, prés du jeton premier joueur. Au centre de la table :
%
\item%
%
 Vous pouvez aussi récupérer des tuiles parmi celles précédemment défaussées.
%
\item%
%
 Choisissez %
\textcolor{mygreen}{%
\textbf{une couleur}%
}%
\textit{ }%
 autre que la %
\textcolor{mygreen}{%
\textbf{couleur joker}%
}%
\textit{ }%
 et ramassez la totalité des tuiles de celle{-}ci sur le cercle.
%
\item%
%
 Si le centre de la table contient une ou plusieurs tuiles de la %
\textcolor{mygreen}{%
\textbf{couleur joker}%
}%
, vous devez en prendre une.
%
\item%
%
 Si vous êtes le premier à récupérer des tuiles au centre de la table, vous devez prendre le jeton 1er joueur et perdre
%
\end{itemize}%
immédiatement au tant de point que de tuiles colorées %
\textcolor{mygreen}{%
\textbf{récupérées}%
}%
\textit{ }%
 au centre de la table. Lorsque vous reculez sur la piste de score, %
\textcolor{mygreen}{%
\textbf{vous ne pouvez pas reculer}%
}%
\textit{ }%
 au{-}delà de la case 01. Vous serez le premier à choisir à la prochaine manche.
%
\begin{description}%
\item[{-} ]%
%
 Cette étape ce termine lorsque les joueurs ont récupérés la %
\textcolor{mygreen}{%
\textbf{totalité des tuiles}%
}%
\textit{ }%
 sur les cercles et au centre de la table.
%
\end{description}

%
\subsection{ PLACER DES TUILES ET MARQUEUR DES POINTS
}%
\label{subsec:PLACERDESTUILESETMARQUEURDESPOINTS}%
\begin{description}%
\item[{-} ]%
%
 À tour de rôle, les joueurs choisissent %
\textcolor{mygreen}{%
\textbf{un espace}%
}%
\textit{ }%
 de l'étoile qu'ils peuvent garnir. Vous devez respecter la couleur et %
\textcolor{mygreen}{%
\textbf{le nombre de tuiles}%
}%
\textit{ }%
 exigés pour remplir une branche de rosace. Vous pouvez utiliser une tuile de la couleur joker pour remplacer une couleur manquante. Vous devez cependant avoir au moins une tuile de la couleur concernée dans le lot.
%
\item[{-} ]%
%
 Si vous remplissez la condition avec vos tuiles, déposez une des tuiles demandées sur la branche de la rosace et %
\textcolor{mygreen}{%
\textbf{défaussez les autres}%
}%
\textit{ }%
 dans la tour de défausse.
%
\item[{-} ]%
%
 L'étoile centrale doit être composée de %
\textcolor{mygreen}{%
\textbf{5 couleurs différentes}%
}%
. Il n'y a pas d'emplacements spécifiques pour les couleurs. Vous devez uniquement respecter le nombre de tuiles exigé sur la branche.
%
\item[{-} ]%
%
 Pour chaque tuile que vous posez sur votre plateau, vous marquez 1 point. Si vous êtes adjacent à une tuiles ou plusieurs connectées vous marquez %
\textcolor{mygreen}{%
\textbf{1 point}%
}%
\textit{ }%
 pour chacune de ces tuiles.
%
\item[{-} ]%
%
 Si vous remplissez les %
\textcolor{mygreen}{%
\textbf{4 emplacements}%
}%
\textit{ }%
 autour d'une case pilier ou statue, vous devez immédiatement récupérer 1 (pilier ) ou 2 (statue) tuiles de votre choix sur l'un des 10 emplacements d'approvisionnement du plateau de score. Si vous remplissez les 2 emplacements autour d'une fenêtre, vous devez immédiatement récupérer 3 tuiles de votre choix. Réapprovisionnez ensuite les branches vides.
%
\item[{-} ]%
%
 À la fin de votre tour, vous pouvez conserver jusqu'à %
\textcolor{mygreen}{%
\textbf{4 tuiles}%
}%
\textit{ }%
 sur votre plateau.%
\textcolor{mygreen}{%
\textbf{ Les tuiles excédentaires}%
}%
\textit{ }%
 sont défaussées dans la tour et vous font perdre %
\textcolor{mygreen}{%
\textbf{1}%
}%
\textit{ }%
 point par tuile %
\textcolor{mygreen}{%
\textbf{défaussée}%
}%
.
%
\end{description}

%
\subsection{ PRÉPARER LA PROCHAINE MANCHE
}%
\label{subsec:PRPARERLAPROCHAINEMANCHE}%
\begin{description}%
\item[{-} ]%
%
 Avancer le marqueur de manche d'une case vers la droite.
%
\item[{-} ]%
%
 Remplissez les cercles de fabrique comme indiqué dans la mise en place. Remettez le marqueur 1er joueur au centre de la table.
%
\end{description}

%
\sectionfont{\color{red}}%
\subsectionfont{\color{red}}%
\subsubsectionfont{\color{red}}%
\section{ End of game
}%
\label{sec:Endofgame}%
\textcolor{red}{\rule{18cm}{0.07cm}}\break%
\textcolor{red}{%
\textbf{À la fin de la sixième manche, la partie prend fin. En plus des points accumulés tout au long de la partie, rajoutez :}%
}%

%
\begin{itemize}%
\item%
%
  %
\textcolor{red}{%
\textbf{14/15/16/17/18/20 points si vous avez entièrement complétée la rosace multicolore/rouge/bleu/jaune/vert/violet.}%
}%

%
\item%
%
 %
\textcolor{red}{%
\textbf{4/8/12/16 points si vous avez complété toutes les branches de valeur 1/2/3/4 des étoiles de votre plateau.}%
}%

%
\end{itemize}%
\begin{description}%
\item[{-} ]%
%
 %
\textcolor{red}{%
\textbf{S'il vous reste des tuiles que vous ne pouvez poser pas à la fin de la sixième manche, vous devez les défausser et perdre autant de points que de tuiles défaussées.}%
}%
\end{description}

%
\end{document}
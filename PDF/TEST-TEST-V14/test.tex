\documentclass{article}%
\usepackage[T1]{fontenc}%
\usepackage[utf8]{inputenc}%
\usepackage{lmodern}%
\usepackage{textcomp}%
\usepackage{lastpage}%
\usepackage{geometry}%
\geometry{margin=0.7in}%
\usepackage{fancyhdr}%
%
\usepackage{sectsty}%
\title{test}%
\author{test}%
\date{\today}%
\fancypagestyle{header}{%
\renewcommand{\headrulewidth}{1pt}%
\renewcommand{\footrulewidth}{1pt}%
\fancyhead{%
}%
\fancyfoot{%
}%
\fancyhead[L]{%
Page \thepage%
}%
\fancyhead[C]{%
QuickBGR%
}%
\fancyhead[R]{%
\today%
}%
\fancyfoot[C]{%
QuickBGR%
}%
}%
%
\begin{document}%
\normalsize%
\maketitle\thispagestyle{header}%
\pagestyle{header}%
\section{ Intro
}%
\label{sec:Intro}%
\rule{18cm}{0.07cm}\break%
Vous participez à la réunion secrète d'une société internationale. Attention, tous les participants ne sont pas des disciples, certains infiltrés ont pris place autour de la table et l'on soupçonne même un journaliste d'être secrètement présent. Grâce à des échanges d'indices autour d'un mot de passe commun, vous allez tenter de démasquer les intrus, sans vous faire éliminer !


%
\section{ Mise en place
}%
\label{sec:Miseenplace}%
\rule{18cm}{0.07cm}\break%
\begin{enumerate}%
\item%
%
 Préparez l'équivalent de %
\textbf{4/5 Missions}%
. Chaque Mission est composée d'une carte %
\textbf{Codes secrets}%
\textit{ }%
 et d'un set de cartes %
\textbf{Disciple}%
,%
\textbf{ Infiltrés}%
\textit{ }%
 et %
\textbf{Journaliste}%
\textit{ }%
 reparties selon le nombre de joueurs.
%
\item%
%
 Choisissez une Mission et distribuez une carte à chaque joueur. Posez au %
\textbf{centre}%
\textit{ }%
 de la table la carte %
\textbf{Codes secrets}%
.
%
\item%
%
 Pour chaque mission, les mots des cartes %
\textbf{Disciples}%
\textit{ }%
 ont la même couleur que ceux de la carte Codes Secrets alors que les cartes Infiltrés ont une couleur %
\textbf{différente}%
.
%
\item%
%
 %
\textbf{Le premier joueur}%
\textit{ }%
 choisit un chiffre entre 1 et 10 pour sélectionner le %
\textbf{Mot de passe}%
\textit{ }%
 sur sa carte.
%
\end{enumerate}

%
\section{ Tour de jeu
}%
\label{sec:Tourdejeu}%
\rule{18cm}{0.07cm}\break%
\textbf{Chacun leur tour, les joueurs annoncent un mot en rapport avec le mot de passe sélectionné. L'infiltré a un mot de passe légérement différent des autres et le journaliste, qui ne connait pas le mot de passe, doit improviser en s'inspirant des mots précédemment annoncés}%
.


%
\subsection{ Donner un indice
}%
\label{subsec:Donnerunindice}%
\begin{description}%
\item[{-} ]%
%
 Chacun leur tour, les joueurs annonce un Indice (un mot) en rapport avec le Mot de passe préalablement sélectionné %
\textbf{indiqué sur leur carte}%
.
%
\item[{-} ]%
%
 A part le Journaliste (qui connait son rôle), les autres joueurs ne savent pas s'ils sont Disciples ou Infiltrés.
%
\end{description}%
\begin{enumerate}%
\item%
%
 %
\textbf{\textit{Les Disciples}}%
\textit{ }%
 : Ils reçoivent tous le %
\textbf{même}%
\textit{ }%
 Mot de passe. Leur objectif est de démasquer les Infiltrés et le Journaliste.
%
\item%
%
 %
\textbf{\textit{Les Infiltrés}}%
\textit{ }%
 : Ils reçoivent un Mot de passe %
\textbf{légérement différent}%
\textit{ }%
 de celui des %
\textit{Disciples}%
. Leur objectif est de se faire passer pour des %
\textit{Disciples}%
.
%
\item%
%
 %
\textbf{\textit{Le Journaliste}}%
\textit{ }%
 : Il ne reçoit %
\textbf{aucun Mot de passe}%
. Son but est de faire croire qu'il en a un, tout en essayant de deviner celui des %
\textit{Disciples}%
.
%
\end{enumerate}

%
\subsection{ Débattre
}%
\label{subsec:Dbattre}%
\begin{description}%
\item[{-} ]%
%
 Une fois que tous les joueurs ont donné leur Indice, débattez entre vous pour déterminer qui peut être le %
\textbf{Journaliste}%
\textit{ }%
 ou un %
\textbf{Infiltré}%
.
%
\item[{-} ]%
%
 Votez tous en même temps en pointant de votre doigt la personne que vous soupçonnez.
%
\item[{-} ]%
%
 Le joueur désigné par la majorité des votants est %
\textbf{éliminé}%
\textit{ }%
 :
%
\end{description}%
\begin{itemize}%
\item%
%
 %
\textbf{Si c’était le Journaliste}%
, il peut encore tenter de deviner le Mot de passe des Disciples. Il l'annonce à haute voix puis vérifie sur la carte Codes secrets. S’il réussit la partie s’arrête et il marque 3 points.S’il échoue, la partie continue.
%
\item%
%
 %
\textbf{Si ce n’était pas le Journaliste}%
, il regarde discrètement la carte Codes secrets et annonce aux autres joueurs s’il était un Disciple (il a le même mot que sur la carte Codes secrets) ou un Infiltré (il a un mot différent).
%
\end{itemize}%
\begin{description}%
\item[{-} ]%
%
 Débutez un nouveau tour en commençant par le joueur à %
\textbf{gauche}%
\textit{ }%
 du joueur éliminé.
%
\item[{-} ]%
%
 Enchaînez ainsi les tours jusqu’à ce que %
\textbf{tous}%
\textit{ }%
 les %
\textit{Infiltrés}%
\textit{ }%
 et le %
\textit{Journaliste}%
\textit{ }%
 aient été éliminés ou qu’il ne reste plus que 2 joueurs :
%
\end{description}%
\begin{itemize}%
\item%
%
 %
\textbf{Les Disciples gagnent}%
\textit{ }%
 si tous les Infiltrés et le Journaliste ont été éliminés, ils marquent alors tous 1 point (y compris les Disciples déjà éliminés).
%
\item%
%
 %
\textbf{Les Infiltrés gagnent}%
\textit{ }%
 si l'un d'eux ne s'est pas fait %
\textbf{\textit{éliminer}}%
\textit{ }%
 , ils marquent alors %
\textbf{tous 2 points}%
\textit{ }%
 (y compris les Infiltrés déjà éliminés).
%
\item%
%
 %
\textbf{Le Journaliste gagne}%
\textit{ }%
 s'il a deviné le mot de passe des Disciples ou s'il n'a pas été éliminé, il marque alors %
\textbf{3 points}%
.
%
\end{itemize}%
\begin{description}%
\item[{-} ]%
%
 Entamez une nouvelle Mission.
%
\end{description}

%
\section{ End of game
}%
\label{sec:Endofgame}%
\rule{18cm}{0.07cm}\break%
\textbf{Le premier joueur atteignant 5 points est déclaré vainqueur}

%
\end{document}
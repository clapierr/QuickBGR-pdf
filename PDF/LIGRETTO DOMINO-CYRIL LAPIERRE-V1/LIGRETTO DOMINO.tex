\documentclass{article}%
\usepackage[T1]{fontenc}%
\usepackage[utf8]{inputenc}%
\usepackage{lmodern}%
\usepackage{textcomp}%
%
\title{LIGRETTO DOMINO}%
\author{Cyril Lapierre}%
\date{\today}%
%
\begin{document}%
\pagestyle{empty}%
\normalsize%
\maketitle%
\section{ Mise en place
}%
\label{sec:Miseenplace}%

%
\subsection{ Plateau central
}%
\label{subsec:Plateaucentral}%
Assemblez le plateau de jeu selon le nombre de joueurs et posez{-}le au centre de la table


%
\subsection{ Chaque joueur reçoit
}%
\label{subsec:Chaquejoueurreoit}%
\begin{itemize}%
\item%
%
 15 tuiles Domino avec le même rebord (simple, double, discontinu ou
%
\end{itemize}%
pointillé). Faites une pile face cachée avec vos 15 tuiles mélangées.
%
\begin{itemize}%
\item%
%
 2 tuiles carrées Joker dont le rebord correspond à celui de vos tuiles Domino
%
\item%
%
 1 mini Plateau joueur rouge
%
\item%
%
 1 pion Ligretto
%
\end{itemize}

%
\section{ Tour de jeu
}%
\label{sec:Tourdejeu}%

%
\subsection{ POSER UN PION LIGRETTO
}%
\label{subsec:POSERUNPIONLIGRETTO}%
Avant chaque manche, les joueurs répartissent ensemble les 4 pions Ligretto sur n’importe quelle case du plateau de jeu.


%
\subsection{ JOUER UNE TUILE DOMINO
}%
\label{subsec:JOUERUNETUILEDOMINO}%
Tous les joueurs jouent simultanément.
%
Chaque joueur doit poser une tuile sur une case vide autour d'un pion Ligretto en remplissant
%
les conditions suivantes :
%
\begin{enumerate}%
\item%
%
 La tuile Domino posée doit être identique aux cases qu'elle recouvre.
%
\item%
%
 Elle ne peut pas recouvrir une tuile Domino déjà posée.
%
\item%
%
 Elle doit être connectée au pion Ligretto par un côté.
%
\end{enumerate}%
Une fois votre tuile Domino posée, déplacez le pion Ligretto sur la couleur la plus éloignée de son
%
emplacement actu el.
%
\begin{itemize}%
\item%
%
 S'il n'existe aucune possibilité de pose à partir de la nouvelle position d'un pion Ligretto ou si vous enfermez un pion
%
\end{itemize}%
Ligretto déjà posé, vous devez poser votre prochaine tuile Ligretto sur n’importe quel emplacement du plateau de jeu
%
qui convient pour les deux couleurs. Posez ensuite immédiatement le pion Ligretto bloqué sur une des deux cases.
%
\begin{itemize}%
\item%
%
 Si vous ne trouvez pas d'emplacement pour poser votre tuile Domino, défaussez{-}la et prenez{-}en une nouvelle.
%
\item%
%
 Si vous désirez poser votre tuile Domino sur un emplacement situé à une case de distance d'un pion Ligretto, vous
%
\end{itemize}%
pouvez recouvrir cette case avec une de vos 2 tuiles Joker. Suite à quoi, seulement vous, pouvez poser une tuile à
%
côté de ce dernier.
%
\begin{itemize}%
\item%
%
 Si vous n'avez plus de tuiles Domino dans votre pile, formez{-}en une nouvelle avec les tuiles de votre défausse .
%
\end{itemize}

%
\subsection{ FIN DE MANCHE
}%
\label{subsec:FINDEMANCHE}%
\begin{itemize}%
\item%
%
 Dès qu’un joueur a posé toutes ses tuiles Domino sur le plateau de jeu, il crie Ligretto Stop et la manche prend fin.
%
\item%
%
 Vérifiez ensemble si toutes les tuiles ont été correctement placées. Si un joueur a commis une erreur lors de la pose,
%
\end{itemize}%
il ne participe pas au décompte de points de la manche.
%
\begin{itemize}%
\item%
%
 Le joueur qui a posé la totalité de ses tuiles Domino reçoit un jeton 2 P.V.
%
\item%
%
 Parmi les joueurs restants, celui qui a le moins de tuiles Domino en sa possession prend un jeton 1 P.V.
%
\item%
%
 Les joueurs qui n'ont pas obtenus de points posent 2 de leurs tuiles Domino sur leur Plateau joueur rouge. Ils partiront
%
\end{itemize}%
avec moins de tuiles lors de la prochaine manche


%
\section{ Fin de partie
}%
\label{sec:Findepartie}%
\textbf{Avant le début de la partie, les joueurs décident du nombre de manches qu’ils souhaitent jouer. Le joueur qui a obtenu le
}%
plus de points à la fin du nombre de manches décidé remporte la partie.%
\textbf{}

%
\end{document}
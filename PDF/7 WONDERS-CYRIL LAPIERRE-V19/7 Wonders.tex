\documentclass{scrartcl}%
\usepackage[T1]{fontenc}%
\usepackage[utf8]{inputenc}%
\usepackage{lmodern}%
\usepackage{textcomp}%
\usepackage{lastpage}%
\usepackage{geometry}%
\geometry{margin=0.7in}%
\usepackage{xcolor}%
\usepackage{fancyhdr}%
%
\usepackage{sectsty}%
\usepackage{graphicx}%
\usepackage{xcolor}%
\definecolor{mygreen}{RGB}{58,170,53}%
\title{\includegraphics[width=8cm,height=4cm,keepaspectratio]{C:/Users/badcy/Documents/Unif/BAC 3/Q1/Projet individuel/2223_INFOB318_QuickBGR/code/my_app/static/img/tmp/Box-7 Wonders.jpg}\break 7 Wonders }%
\author{Cyril Lapierre}%
\date{\today \break Tags: Ancient, Card Game, City Building, Civilization, Economic}%
\fancypagestyle{header}{%
\renewcommand{\headrulewidth}{1pt}%
\renewcommand{\footrulewidth}{1pt}%
\fancyhead{%
}%
\fancyfoot{%
}%
\fancyhead[L]{%
Page \thepage%
}%
\fancyhead[C]{%
\includegraphics[width=4cm,height=1cm,keepaspectratio]{C:/Users/badcy/Documents/Unif/BAC 3/Q1/Projet individuel/2223_INFOB318_QuickBGR/code/my_app/static/img/logo/QuickBGR-black-green.png}%
}%
\fancyhead[R]{%
\today%
}%
\fancyfoot[C]{%
\includegraphics[width=4cm,height=1cm,keepaspectratio]{C:/Users/badcy/Documents/Unif/BAC 3/Q1/Projet individuel/2223_INFOB318_QuickBGR/code/my_app/static/img/logo/QuickBGR-black-green.png}%
}%
}%
%
\begin{document}%
\normalsize%
\maketitle\thispagestyle{header}%
\pagestyle{header}%
\sectionfont{\color{blue}}%
\subsectionfont{\color{blue}}%
\subsubsectionfont{\color{blue}}%
\section{ Intro
}%
\label{sec:Intro}%
\textcolor{blue}{\rule{18cm}{0.07cm}}\break%
You have 3 ages to develop a great city of the ancient world and build one of the 7 wonders of the
%
world. using your cards, develop the military or scientific side of your city. Highlight
%
your city by constructing prestigious buildings, without forgetting to develop your economy.
%
Manage your city as well as possible and leave a trace in history...


%
\sectionfont{\color{mygreen}}%
\subsectionfont{\color{mygreen}}%
\subsubsectionfont{\color{mygreen}}%
\section{ Establishment
}%
\label{sec:Establishment}%
\textcolor{mygreen}{\rule{18cm}{0.07cm}}\break%
%
\begin{center}\includegraphics[width=6cm,height=6cm,keepaspectratio]{C:/Users/badcy/Documents/Unif/BAC 3/Q1/Projet individuel/2223_INFOB318_QuickBGR/code/my_app/static/img/tmp/1.jpg}\end{center}%

%
\begin{enumerate}%
\item%
%
 Remove the %
\textcolor{mygreen}{%
\textbf{unused cards}%
}%
, according to the number of players (number at the bottom of the card).
%
\item%
%
 %
\textcolor{mygreen}{%
\textbf{Guild}%
}%
: In the deck of %
\textcolor{mygreen}{%
\textbf{Age 3}%
}%
: add the number of Guilds (8 purple face{-}up cards) according to the number of players (number of players + 2).
%
\item%
%
 Make a pile with the 3 kinds of %
\textcolor{mygreen}{%
\textbf{Military tokens}%
}%
\textit{ }%
 (red).
%
\end{enumerate}%
%
\begin{center}\includegraphics[width=4cm,height=4cm,keepaspectratio]{C:/Users/badcy/Documents/Unif/BAC 3/Q1/Projet individuel/2223_INFOB318_QuickBGR/code/my_app/static/img/tmp/2.jpg}\end{center}%

%
Each player receives:
%
\begin{itemize}%
\item%
%
 1 Wonder card
%
\item%
%
 3 gold coins%
\includegraphics[width=1cm,height=1cm,keepaspectratio]{C:/Users/badcy/Documents/Unif/BAC 3/Q1/Projet individuel/2223_INFOB318_QuickBGR/code/my_app/static/img/tmp/3.jpg}%

%
\item%
%
 7 Age 1 cards
%
\end{itemize}

%
\sectionfont{\color{red}}%
\subsectionfont{\color{red}}%
\subsubsectionfont{\color{red}}%
\section{ Game turn
}%
\label{sec:Gameturn}%
\textcolor{red}{\rule{18cm}{0.07cm}}\break%
%
\begin{center}\includegraphics[width=2cm,height=2cm,keepaspectratio]{C:/Users/badcy/Documents/Unif/BAC 3/Q1/Projet individuel/2223_INFOB318_QuickBGR/code/my_app/static/img/tmp/4.jpg}\end{center}%

%
A game is played in 3 ages represented by 3 stacks of cards numbered I, II and III. For each age, do the following:


%
\subsection{ CHOOSE A CARD
}%
\label{subsec:CHOOSEACARD}%
\begin{description}%
\item[{-} ]%
%
 Choose 1 %
\textcolor{red}{%
\textbf{card}%
}%
\textit{ }%
 from the cards in your hand.
%
\end{description}

%
\subsection{ USING THE CHOSEN CARD:
}%
\label{subsec:USINGTHECHOSENCARD}%
There are 3 possible actions with this card:
%
\begin{description}%
\item[{-} ]%
%
 %
\textcolor{red}{%
\textbf{BUILD THE BUILDING}%
}%
:
%
\end{description}%
\begin{itemize}%
\item%
%
 The construction cost is indicated by a number of resources to be spent (on the left of the map). If the area is %
\textcolor{red}{%
\textbf{empty}%
}%
, the card is free.
%
\item%
%
 You must have the necessary resources in your City. Resources are generated using the Production cards (brown and gray) that you have previously placed.
%
\item%
%
 If you already have a Building with the same %
\textcolor{red}{%
\textbf{symbol}%
}%
\textit{ }%
 as the one shown next to the cost, you pay nothing (Chaining).
%
\item%
%
 If you lack resources, you can buy them from your neighbors (right and left) by %
\textit{Trading}%
. In this case, give 2 coins to this neighbor to use the resource(s) produced by one of his brown or gray cards.
%
\item%
%
 You cannot build 2 identical Buildings (same name).
%
\item%
%
 Place the card above your Wonder.
%
\end{itemize}%
\begin{description}%
\item[{-} ]%
%
 %
\textcolor{red}{%
\textbf{BUILD A STAGE OF HIS WONDER}%
}%
:
%
\end{description}%
\begin{itemize}%
\item%
%
 If you have the necessary resources: cost indicated on each location of the Wonders.
%
\item%
%
 Ability to Trade with your right and left neighbors to obtain missing resources. Slide the card %
\textit{back}%
\textit{ }%
 side up under the leftmost %
\textcolor{red}{%
\textbf{free}%
}%
\textit{ }%
 slot of your Wonder.
%
\end{itemize}%
\begin{description}%
\item[{-} ]%
%
 %
\textcolor{red}{%
\textbf{DISCARD THE CARD TO GET 3 GOLD COINS.}%
}%

%
\end{description}

%
\subsection{ PASS THE REMAINING CARDS
}%
\label{subsec:PASSTHEREMAININGCARDS}%
\begin{description}%
\item[{-} ]%
%
 Pass the remaining cards in your hand to your neighbor: left (1st and 3rd Age)/right (2nd Age).
%
\item[{-} ]%
%
 On the sixth turn, choose a card from the 2 cards you have just received and discard the other. You don't get any coins for the discarded card this way.
%
\end{description}

%
\subsection{ END OF AN AGE
}%
\label{subsec:ENDOFANAGE}%
\begin{description}%
\item[{-} ]%
%
 Resolve Military Conflicts: Compare the number of %
\textcolor{red}{%
\textbf{Shields}%
}%
\textit{ }%
 in your City with those of your right and left neighbors.
%
\item[{-} ]%
%
 If you have a total higher than that of a neighboring City, gain 1 Military token (Age 1 = 1 point / Age 2 = 3 points / Age 3 = 5 points) or {-}1 token (%
\textit{{-}1 point}%
\textit{ }%
 ) if your total is lower. In the event of a tie, no token is taken.
%
\end{description}%
\textcolor{red}{%
\textbf{CARDS DESCRIPTION}%
}%

%
\begin{description}%
\item[{-} ]%
%
 %
\textcolor{red}{%
\textbf{Brown cards}%
}%
: Raw materials. They produce the resources shown on the card each turn.
%
\item[{-} ]%
%
 %
\textcolor{red}{%
\textbf{Grey cards}%
}%
: Manufactured products. They produce scarcer resources.
%
\item[{-} ]%
%
 %
\textcolor{red}{%
\textbf{Blue cards}%
}%
: Civilian buildings: They are worth victory points at the end of the game.
%
\item[{-} ]%
%
 %
\textcolor{red}{%
\textbf{Red Cards}%
}%
: Military Buildings: Increase Military Strength (Shields). Allows you to win militarily over your neighbors and gain victory points during End of Age conflicts.
%
\item[{-} ]%
%
 %
\textcolor{red}{%
\textbf{Yellow cards}%
}%
: Commercial buildings: Produce money and commercial advantages. (More advantageous trade with neighbors or additional production of resources).
%
\item[{-} ]%
%
 %
\textcolor{red}{%
\textbf{Purple cards}%
}%
: Guilds: Allow you to earn special points at the end of the game.%
\includegraphics[width=2cm,height=2cm,keepaspectratio]{C:/Users/badcy/Documents/Unif/BAC 3/Q1/Projet individuel/2223_INFOB318_QuickBGR/code/my_app/static/img/tmp/5.jpg}%

%
\item[{-} ]%
%
 %
\textcolor{red}{%
\textbf{Green cards}%
}%
: Scientific buildings: Allow you to score points according to the number of symbols. The 2 ways of scoring are cumulative:
%
\end{description}%
\begin{itemize}%
\item%
%
 Identical symbols = total multiplied by square.
%
\item%
%
 3 different symbols = 7 points
%
\end{itemize}

%
\sectionfont{\color{cyan}}%
\subsectionfont{\color{cyan}}%
\subsubsectionfont{\color{cyan}}%
\section{ End of Game
}%
\label{sec:EndofGame}%
\textcolor{cyan}{\rule{18cm}{0.07cm}}\break%
The game ends at the end of the 3rd Age. Count your victory points. The player with the most wins the game:
%
\begin{enumerate}%
\item%
%
 Add up the Military Victory Points.
%
\item%
%
 Each set of 3 coins = 1 victory point.
%
\item%
%
 Victory points of the different combinations of %
\textcolor{cyan}{\textbf{\textit{Scientific}}}%
\textit{ }%
 Buildings.
%
\end{enumerate}%
Add the victory points indicated on:
%
\begin{enumerate}%
\item%
%
 %
\textit{Wonder Cards}%
.
%
\item%
%
 %
\textit{Civilian Buildings}%
.
%
\item%
%
 %
\textit{Special yellow cards}%
.
%
\item%
%
 %
\textit{Guild Cards}%
.%
\end{enumerate}

%
\end{document}
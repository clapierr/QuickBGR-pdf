\documentclass{scrartcl}%
\usepackage[T1]{fontenc}%
\usepackage[utf8]{inputenc}%
\usepackage{lmodern}%
\usepackage{textcomp}%
\usepackage{lastpage}%
\usepackage{geometry}%
\geometry{margin=0.7in}%
\usepackage{fancyhdr}%
%
\usepackage{sectsty}%
\usepackage{graphicx}%
\title{\includegraphics[width=8cm,height=4cm,keepaspectratio]{C:/Users/badcy/Documents/Unif/BAC 3/Q1/Projet individuel/2223_INFOB318_QuickBGR/code/my_app/static/img/tmp/Box-Vikings Saga.jpg}\break Vikings Saga }%
\author{Cyril Lapierre NC}%
\date{\today \break Tag: Mythology}%
\fancypagestyle{header}{%
\renewcommand{\headrulewidth}{1pt}%
\renewcommand{\footrulewidth}{1pt}%
\fancyhead{%
}%
\fancyfoot{%
}%
\fancyhead[L]{%
Page \thepage%
}%
\fancyhead[C]{%
\includegraphics[width=4cm,height=1cm,keepaspectratio]{C:/Users/badcy/Documents/Unif/BAC 3/Q1/Projet individuel/2223_INFOB318_QuickBGR/code/my_app/static/img/logo/QuickBGR-NB.png}%
}%
\fancyhead[R]{%
\today%
}%
\fancyfoot[C]{%
\includegraphics[width=4cm,height=1cm,keepaspectratio]{C:/Users/badcy/Documents/Unif/BAC 3/Q1/Projet individuel/2223_INFOB318_QuickBGR/code/my_app/static/img/logo/QuickBGR-NB.png}%
}%
}%
%
\begin{document}%
\normalsize%
\maketitle\thispagestyle{header}%
\pagestyle{header}%
\section{ Mise en place
}%
\label{sec:Miseenplace}%
\rule{18cm}{0.07cm}\break

%
\subsection{ Plateau
}%
\label{subsec:Plateau}%
\begin{description}%
\item[{-} ]%
%
 Posez le Sentier des Vikings au début de la piste numérotée du plateau. Faites coïncider les %
\textbf{ombres}%
\textit{ }%
 du sentier avec les piliers en bois illustrés sur le plateau.
%
\item[{-} ]%
%
 Pour vos premières parties, posez les 9 cartes Aventures suivantes au dessus du plateau :
%
\end{description}%
\begin{itemize}%
\item%
%
 La cité perdue 
%
\item%
%
 L’île trompeuse 
%
\item%
%
 La vielle ville des nains 
%
\item%
%
 La porte aux runes 
%
\item%
%
 Les gardiens en pierre 
%
\item%
%
 LE DÉSERT DE LA VALLÉE DE LA GLACE.
%
\item%
%
 LES SORCIÈRES DE LA MER 
%
\item%
%
 Le pont arc en ciel Bifröst (x2).
%
\end{itemize}%
\begin{description}%
\item[{-} ]%
%
 Récupérez les cartes Vikings pour chaque %
\textbf{couleur}%
\textit{ }%
 de carte (sauf cartes bleues et noires). Prenez pour chaque type de %
\textbf{Viking}%
, autant de cartes que de joueur. Déposez{-}les sur les cartes Aventure concernées.
%
\item[{-} ]%
%
 Prenez les cartes Déplacement spécial correspondant aux cartes Aventures choisies à l'aide du %
\textbf{code}%
\textit{ }%
 inscrit dans l'angle inférieur droit de la carte. Déposez{-}les sur les cartes Aventures concernées.%
\includegraphics[width=2cm,height=2cm,keepaspectratio]{C:/Users/badcy/Documents/Unif/BAC 3/Q1/Projet individuel/2223_INFOB318_QuickBGR/code/my_app/static/img/tmp/1.jpg}%

%
\item[{-} ]%
%
 Prenez les jetons Marqueur d'étape :
%
\end{description}%
\begin{itemize}%
\item%
%
 Parmi les 5 jetons %
\textbf{Marqueurs}%
\textit{ }%
 d'étape foncés, prenez{-}en 3 au hasard et posez{-}les sur les 2 cartes noires Bifröst.
%
\item%
%
 Répartissez les 27 jetons Marqueur d'étape clairs sur les 7 autres cartes Déplacement (3x7).
%
\end{itemize}%
\begin{description}%
\item[{-} ]%
%
 Formez une réserve avec les pièces d'or et les 4 jetons Prophétie .
%
\item[{-} ]%
%
 Mélangez les cartes Dieux et formez une pioche près du plateau de jeu.
%
\item[{-} ]%
%
 Mélangez les 14 cartes Déplacement et formez une pioche en bas à gauche sous le plateau de jeu. 
%
\end{description}

%
\subsection{ Chaque joueur reçoit
}%
\label{subsec:Chaquejoueurreoit}%
%
\begin{center}\includegraphics[width=5cm,height=5cm,keepaspectratio]{C:/Users/badcy/Documents/Unif/BAC 3/Q1/Projet individuel/2223_INFOB318_QuickBGR/code/my_app/static/img/tmp/2.jpg}\end{center}%

%
\begin{description}%
\item[{-} ]%
%
 1 %
\textbf{Bouclier}%
\textit{ }%
 Valhalla à sa couleur.
%
\item[{-} ]%
%
 7 cartes Viking. Mélangez{-}les et posez{-}kes à gauche du Bouclier, face cachée.
%
\item[{-} ]%
%
 1 %
\textbf{pion}%
\textit{ }%
 Viking de sa couleur. Posez{-}le sur la 1ère case du Sentier des Vikings.
%
\item[{-} ]%
%
 1 %
\textbf{jeton}%
\textit{ }%
 de victoire. Posez{-}le sur la case 5 de la piste des Points de Victoire.
%
\item[{-} ]%
%
 5 pièces d'or
%
\end{description}

%
\subsection{ Carte Aventure
}%
\label{subsec:CarteAventure}%
\begin{description}%
\item[{-} ]%
%
 Lisez à haute voix le texte de la %
\textbf{première}%
\textit{ }%
 carte Aventure mise de côté au{-}dessus du plateau.
%
\item[{-} ]%
%
 Retournez la carte et posez{-}la sur la table de façon à ce que les chiffres recouvrent ceux du plateau.
%
\item[{-} ]%
%
 Déposez les 3 jetons %
\textbf{Marqueur d'étape (face visible)}%
\textit{ }%
 sur les cases du plateau avec les chiffres correspondants à ceux indiqués sur la carte Aventure.
%
\item[{-} ]%
%
 Prenez les cartes Viking mises de côté avec cette carte Aventure et séparez les différents Vikings en piles distinctes près du plateau de jeu. Vous pourrez les acquérir durant la partie.
%
\item[{-} ]%
%
 Mélangez les cartes Déplacement spécial correspondant à la %
\textbf{carte Aventure}%
\textit{ }%
 en cours. Mélangez{-}les dans la pioche des cartes %
\textbf{Déplacement}%
. 
%
\end{description}

%
\section{ Tour de jeu
}%
\label{sec:Tourdejeu}%
\rule{18cm}{0.07cm}\break%
À chaque manche, vous jouez une carte Aventure différente. Chaque manche se déroule en 5 étapes :


%
\subsection{ ACHAT DE NOUVEAUX VIKINGS (SAUF DERNIÈRE MANCHE)
}%
\label{subsec:ACHATDENOUVEAUXVIKINGS(SAUFDERNIREMANCHE)}%
\begin{description}%
\item[{-} ]%
%
 Vous pouvez chacun votre tour, acheter 1%
\textbf{ nouvelle}%
\textit{ }%
 carte Viking parmi celles disponibles (cartes Aventure en cours et précédemment jouées).%
\textbf{ Une fois le tour achevé}%
, vous pouvez de nouveau acheter un Viking disponible, etc.
%
\item[{-} ]%
%
 Mélangez les cartes %
\textbf{Viking}%
\textit{ }%
 achetées avec celles de votre pioche.
%
\end{description}

%
\subsection{ PIOCHER DES CARTES
}%
\label{subsec:PIOCHERDESCARTES}%
\begin{description}%
\item[{-} ]%
%
 Selon votre position sur la piste de victoire (1er/2nd/3ème/4ème), vous devez piochez 3/4/5/6 cartes dans votre pioche de cartes Viking.
%
\end{description}

%
\subsection{ DÉPLACEMENT DES VIKINGS
}%
\label{subsec:DPLACEMENTDESVIKINGS}%
\begin{description}%
\item[{-} ]%
%
 Lisez à haute voix le texte de la première carte Aventure mise de côté au{-}dessus du plateau,
%
\item[{-} ]%
%
 Retournez la carte et posez{-}la sur la table de façon à ce que les chiffres recouvrent ceux du plateau.
%
\item[{-} ]%
%
 Déposez les %
\textbf{3 jetons Marqueur}%
\textit{ }%
 d'étape (face visible) sur les cases du plateau avec les chiffres correspondants à ceux indiqués sur la carte Aventure.
%
\item[{-} ]%
%
 Prenez les cartes Viking mises de côté avec cette carte %
\textbf{Aventure}%
\textit{ }%
 et séparez les différents Vikings en piles distinctes près du plateau de jeu. Vous pourrez les acquérir durant la partie.
%
\item[{-} ]%
%
 Mélangez les cartes Déplacement spécial correspondant à la carte Aventure en cours. Mélangez{-}les dans la pioche des cartes Déplacement. %
\textbf{Vous dirigez un petit}%
\textit{ }%
 groupe de Vikings que vous améliorerez au fil des aventures en embauchant de%
\textbf{ nouveaux combattants}%
\textit{ }%
 aux pouvoirs spécifiques. Chaque aventure est différente et offre son lot de primes et de sanctions. Avancez précautionneusement sur le sentier des Vikings pour récupérer des bonus et éviter les pénalités. Calculez bien votre avancement pour finir sur les meilleures cases de la carte aventure en cours et finir votre périple en beauté.
%
\end{description}%
\begin{itemize}%
\item%
%
 Prenez le nombre de %
\textbf{pièces d'or}%
\textit{ }%
 indiqué sur la pièce.
%
\item%
%
 Rendez le nombre de %
\textbf{pièces d'or}%
\textit{ }%
 indiqué sur la pièce.
%
\item%
%
 Retirez définitivement une carte Viking que vous avez %
\textbf{déjà jouée}%
\textit{ }%
 au cours de cette Aventure.
%
\item%
%
 Pour chaque Casque, déplacez votre pion P.V sur la piste des %
\textbf{Points de Victoire}%
.
%
\item%
%
 Pour %
\textbf{chaque Casque brisé}%
, reculez votre pion P.V sur la piste des Points de Victoire.
%
\item%
%
 Vous pouvez prendre %
\textbf{gratuitement}%
\textit{ }%
 n'importe quelle carte Viking disponible sur le Marché (déplacement en cours ou passée).
%
\end{itemize}%
%
\begin{center}\includegraphics[width=4cm,height=4cm,keepaspectratio]{C:/Users/badcy/Documents/Unif/BAC 3/Q1/Projet individuel/2223_INFOB318_QuickBGR/code/my_app/static/img/tmp/3.jpg}\end{center}%

%
\begin{description}%
\item[{-} ]%
%
 Si votre pion Viking atteint ou dépasse la dernière case de la carte Déplacement, vous subissez la pénalité
%
\end{description}%
illustrée sur la dernière case. Si en plus vous quittez le Sentier des Vikings, vous subissez la pénalité une seconde
%
fois. L’ Aventure est automatiquement terminée.
%
\begin{description}%
\item[{-} ]%
%
 Prenez ensuite une (et une seule) carte Viking de votre pioche pour remplacer les cartes jouées.
%
\item[{-} ]%
%
 Si vous n'avez pas encore atteint la carte Aventure, vous continuez à jouer.
%
\item[{-} ]%
%
 Si vous avez atteint la carte Aventure, vous pouvez au choix :
%
\end{description}%
\begin{itemize}%
\item%
%
 Poursuivre l'Aventure.
%
\item%
%
 Quitter l'Aventure.
%
\end{itemize}%
\begin{description}%
\item[{-} ]%
%
 Cette étape prend fin quand tous les joueurs ont quitté l'Aventure.
%
\end{description}

%
\subsection{ VALHALLA
}%
\label{subsec:VALHALLA}%
\begin{description}%
\item[{-} ]%
%
 Vous pouvez déposer une des cartes Viking (pas Dieu) que vous avez jouée sous votre Bouclier. Votre combattant est envoyé au Valhalla et fera partie du paquet de cartes que vous devrez utilisez lors la dernière Aventure Pont arc en ciel %
\textbf{\textit{Bifröst}}%
.
%
\item[{-} ]%
%
 En fin de partie, vous recevrez 1 point de victoire pour chaque Casque %
\textbf{présent}%
\textit{ }%
 sur les cartes des Vikings envoyés au Valhalla.
%
\end{description}

%
\subsection{ PRÉPARER LA NOUVELLE AVENTURE
}%
\label{subsec:PRPARERLANOUVELLEAVENTURE}%
%
\begin{center}\includegraphics[width=5cm,height=5cm,keepaspectratio]{C:/Users/badcy/Documents/Unif/BAC 3/Q1/Projet individuel/2223_INFOB318_QuickBGR/code/my_app/static/img/tmp/4.jpg}\end{center}%

%
\begin{description}%
\item[{-} ]%
%
 Rangez la carte Aventure qui vient d'être jouée, ainsi que les %
\textbf{Marqueurs d'étape}%
\textit{ }%
 utilisés.
%
\item[{-} ]%
%
 Les cartes Déplacement précédemment jouées ne sont pas remises dans la pioche.
%
\item[{-} ]%
%
 Les cartes Dieux jouées sont remises sous la %
\textbf{pioche}%
\textit{ }%
 concernée.
%
\item[{-} ]%
%
 Changez %
\textbf{de 1er joueur.}%

%
\item[{-} ]%
%
 Prenez la prochaine carte Aventure, lisez{-}la %
\textbf{à haute voix}%
. Posez ensuite la carte sur le plateau avec les 3 marqueurs qui l'accompagnent.
%
\item[{-} ]%
%
 Retournez la carte et posez{-}la sur la table de façon à ce que les chiffres recouvrent ceux du plateau.
%
\item[{-} ]%
%
 Déposez les 3 jetons Marqueur d'étape (face visible) sur les cases du plateau qui correspondent aux chiffres indiqués sur la carte Aventure.
%
\item[{-} ]%
%
 Prenez les cartes Viking mises de côté avec cette carte %
\textbf{Aventure}%
\textit{ }%
 et séparez les différents Vikings en piles distinctes près du plateau de jeu.
%
\item[{-} ]%
%
 Mélangez %
\textbf{les cartes Déplacement spécial}%
\textit{ }%
 correspondant à la %
\textbf{carte Déplacement}%
\textit{ }%
 en cours. %
\textbf{Mélangez{-}les}%
\textit{ }%
 dans la pioche des cartes Déplacement.
%
\item[{-} ]%
%
 Récupérez vos cartes jouées et celles de votre main et mélangez{-}les à votre pioche de carte Viking.
%
\item[{-} ]%
%
 %
\textbf{Aventure en Mer }%
: Si vous devez jouer des cartes bleues, vous devrez utiliser le Bateau Viking à la place du Sentier des Vikings. Il s'utilise de la même manière que le Sentier.
%
\item[{-} ]%
%
 %
\textbf{La dernière Aventure}%
\textit{ }%
 (Pont Arc en Ciel Bifröst) :
%
\end{description}%
\begin{itemize}%
\item%
%
 Vous jouez 2 cartes Aventure que vous %
\textbf{disposez}%
\textit{ }%
 selon les chiffres indiqués sur les cartes.
%
\item%
%
 Comptez le nombre de cartes Viking et Dieux %
\textbf{que}%
\textit{ }%
 vous n'avez pas glissé sous votre Bouclier Valhalla. Recevez %
\textbf{1 pièce d'or}%
\textit{ }%
 par carte.
%
\item%
%
 Récupérez les cartes préalablement glissées sous %
\textbf{votre Bouclier}%
. Elles constituent votre nouvelle pioche de cartes Viking.
%
\item%
%
 Vous ne pourrez quitter l'Aventure que lorsque votre pion aura atteint la seconde carte Aventure Bifröst.
%
\end{itemize}

%
\section{ Fin de partie
}%
\label{sec:Findepartie}%
\rule{18cm}{0.07cm}\break%
\textbf{La partie se termine après avoir fini la dernière Aventure Bifröst.}%

%
\begin{description}%
\item[{-} ]%
%
 %
\textbf{Avancez d'1 point de victoire sur la piste pour chaque lot de 3 pièces en votre possession.}%

%
\item[{-} ]%
%
 %
\textbf{Comptez le nombre de Casques que vous avez sur les cartes Vikings utilisées pour la dernière aventure (cartes Valhalla).}%

%
\item[{-} ]%
%
 %
\textbf{Le joueur le plus avancé sur la piste gagne la partie. En cas d'égalité, celui qui possède le plus d'or remporte la partie.}%

%
\end{description}%
%
\begin{center}\includegraphics[width=10cm,height=10cm,keepaspectratio]{C:/Users/badcy/Documents/Unif/BAC 3/Q1/Projet individuel/2223_INFOB318_QuickBGR/code/my_app/static/img/tmp/5.jpg}\end{center}%

%
\end{document}
\documentclass{scrartcl}%
\usepackage[T1]{fontenc}%
\usepackage[utf8]{inputenc}%
\usepackage{lmodern}%
\usepackage{textcomp}%
\usepackage{lastpage}%
\usepackage{geometry}%
\geometry{margin=0.7in}%
\usepackage{xcolor}%
\usepackage{fancyhdr}%
%
\usepackage{sectsty}%
\usepackage{graphicx}%
\usepackage{xcolor}%
\definecolor{mygreen}{RGB}{58,170,53}%
\title{\includegraphics[width=8cm,height=4cm,keepaspectratio]{C:/Users/badcy/Documents/Unif/BAC 3/Q1/Projet individuel/2223_INFOB318_QuickBGR/code/my_app/static/img/tmp/Box-Small World.jpg}\break Small World }%
\author{Cyril Lapierre}%
\date{\today \break Tags: Fantasy, Fighting, Territory Building}%
\fancypagestyle{header}{%
\renewcommand{\headrulewidth}{1pt}%
\renewcommand{\footrulewidth}{1pt}%
\fancyhead{%
}%
\fancyfoot{%
}%
\fancyhead[L]{%
Page \thepage%
}%
\fancyhead[C]{%
\includegraphics[width=4cm,height=1cm,keepaspectratio]{C:/Users/badcy/Documents/Unif/BAC 3/Q1/Projet individuel/2223_INFOB318_QuickBGR/code/my_app/static/img/logo/QuickBGR-black-green.png}%
}%
\fancyhead[R]{%
\today%
}%
\fancyfoot[C]{%
\includegraphics[width=4cm,height=1cm,keepaspectratio]{C:/Users/badcy/Documents/Unif/BAC 3/Q1/Projet individuel/2223_INFOB318_QuickBGR/code/my_app/static/img/logo/QuickBGR-black-green.png}%
}%
}%
%
\begin{document}%
\normalsize%
\maketitle\thispagestyle{header}%
\pagestyle{header}%
\sectionfont{\color{blue}}%
\subsectionfont{\color{blue}}%
\subsubsectionfont{\color{blue}}%
\section{ Mise en place
}%
\label{sec:Miseenplace}%
\textcolor{blue}{\rule{18cm}{0.07cm}}\break%
%
\begin{center}\includegraphics[width=6cm,height=5cm,keepaspectratio]{C:/Users/badcy/Documents/Unif/BAC 3/Q1/Projet individuel/2223_INFOB318_QuickBGR/code/my_app/static/img/tmp/1.jpg}\end{center}%

%

%
\begin{description}%
\item[{-} ]%
%
 Placez le bon plateau sur la table selon le nombre de joueurs. 
%
\item[{-} ]%
%
 Placez %
\textcolor{blue}{%
\textbf{le jeton}%
}%
\textit{ }%
 compte{-}tours (couronne) sur la première case de la piste compte{-}tours
%
\item[{-} ]%
%
 Mélangez toutes les tuiles Peuple et tirez en 5 au hasard. Placez{-}les les unes sous les autres, face colorée visible. Le reste forme une pioche face visible, placée sous les 5 tuiles.
%
\item[{-} ]%
%
 Mélangez toutes les tuiles Pouvoir et tirez en 5 au hasard. Placez{-}les, face visible, à droite de chaque tuile de Peuple en les imbriquant dans celles{-}ci. Le reste forme une pioche imbriquée dans la pioche Peuple.
%
\item[{-} ]%
%
 Placez un pion %
\textcolor{blue}{%
\textbf{Tribu}%
}%
\textit{ }%
 oubliée sur chacune des régions de la carte portant le symbole correspondant.
%
\item[{-} ]%
%
 Placez un pion Montagne sur chacune des régions montagneuses de la carte.
%
\item[{-} ]%
%
 Chaque joueur commence avec %
\textcolor{blue}{%
\textbf{5 jetons Victoire}%
}%
\textit{ }%
 de valeur %
\textcolor{blue}{%
\textbf{1}%
}%
.%
\includegraphics[width=2cm,height=3cm,keepaspectratio]{C:/Users/badcy/Documents/Unif/BAC 3/Q1/Projet individuel/2223_INFOB318_QuickBGR/code/my_app/static/img/tmp/2.jpg}%

%

%
\end{description}

%
\sectionfont{\color{mygreen}}%
\subsectionfont{\color{mygreen}}%
\subsubsectionfont{\color{mygreen}}%
\section{ Tour de jeu
}%
\label{sec:Tourdejeu}%
\textcolor{mygreen}{\rule{18cm}{0.07cm}}\break%
\begin{description}%
\item[{-} ]%
%
 Lors de son tour de jeu, chaque joueur effectue 3 actions dans l'ordre suivant.
%
\end{description}

%
\subsection{ CHOISIR UNE COMBINAISON PEUPLE {-} POUVOIR SPÉCIAL (FACULTATIF) :
}%
\label{subsec:CHOISIRUNECOMBINAISONPEUPLE{-}POUVOIRSPCIAL(FACULTATIF)}%
%
\includegraphics[width=3cm,height=3cm,keepaspectratio]{C:/Users/badcy/Documents/Unif/BAC 3/Q1/Projet individuel/2223_INFOB318_QuickBGR/code/my_app/static/img/tmp/3.jpg}%
%
\includegraphics[width=3cm,height=3cm,keepaspectratio]{C:/Users/badcy/Documents/Unif/BAC 3/Q1/Projet individuel/2223_INFOB318_QuickBGR/code/my_app/static/img/tmp/4.jpg}%

%
%
\includegraphics[width=3cm,height=3cm,keepaspectratio]{C:/Users/badcy/Documents/Unif/BAC 3/Q1/Projet individuel/2223_INFOB318_QuickBGR/code/my_app/static/img/tmp/5.jpg}%

%

%
\begin{description}%
\item[{-} ]%
%
 Le joueur dont c'est le tour choisit une combinaison Peuple/Pouvoir parmi les six combinaisons visibles.
%
\item[{-} ]%
%
 La combinaison placée tout en haut de la colonne est gratuite. Ensuite, payez 1 jeton victoire sur chaque combinaison de tuiles précédant la combinaison que vous voulez prendre.
%
\item[{-} ]%
%
 Quand un joueur choisit une %
\textcolor{mygreen}{%
\textbf{combinaison}%
}%
, il récupère autant de pions Peuple correspondants que
%
\end{description}%
la somme des valeurs indiquées sur les %
\textcolor{mygreen}{%
\textbf{2 tuiles de sa combinaison}%
}%
\textit{ }%
 (sur fond orange). Il récupère aussi les éventuels jetons victoire posés sur sa combinaison.
%
\begin{description}%
\item[{-} ]%
%
 Décalez les tuiles vers le haut pour remplir l’espace laissé par votre choix. La %
\textcolor{mygreen}{%
\textbf{combinaison}%
}%
\textit{ }%
 en haut de la pioche remonte et dévoile une nouvelle combinaison. Il y a donc toujours 6 choix parmi les Peuples disponibles.
%
\end{description}

%
\subsection{ CONQUÉRIR DES RÉGIONS
}%
\label{subsec:CONQURIRDESRGIONS}%
\begin{description}%
\item[{-} ]%
%
 %
\textcolor{mygreen}{%
\textbf{Première conquête}%
}%
\textit{ }%
 : Lorsqu'un nouveau Peuple débarque sur le plateau, il doit obligatoirement arriver par un territoire en bordure de plateau ou touchant une mer elle{-}même en bordure de plateau.
%
\item[{-} ]%
%
 %
\textcolor{mygreen}{%
\textbf{Conquêtes suivantes}%
}%
\textit{ }%
 : Pour envahir une région, un joueur doit déployer 2 pions de Peuple ET :
%
\end{description}%
\begin{itemize}%
\item%
%
 1 pion de Peuple supplémentaire par Campement, Antre de Troll, Forteresse ou Montagne.
%
\item%
%
 1 pion de Peuple supplémentaire par Tribu oubliée.
%
\item%
%
 1 pion de Peuple supplémentaire par pion de Peuple ennemi présent dans cette région.
%
\end{itemize}%
\begin{description}%
\item[{-} ]%
%
 Les mers et le lac ne peuvent pas être conquis.
%
\item[{-} ]%
%
 Une fois qu'un joueur a déplacé ses jetons Peuple sur un territoire, il ne les bouge plus jusqu'à la phase Redéploiement.
%
\item[{-} ]%
%
 Toute nouvelle conquête doit être adjacente à la région d'où vient le Peuple attaquant.
%
\item[{-} ]%
%
 %
\textcolor{mygreen}{%
\textbf{Dernière conquête}%
}%
: Le joueur peut décider avec les dernières tuiles à jouer (minimum 1) de conquérir un territoire dont les jetons ennemis sont plus nombreux. Il peut s'aider du dé de Renfort. Les points obtenus sur le dé rajoutent autant de renfort à ses unités déjà engagées.
%
\item[{-} ]%
%
 %
\textcolor{mygreen}{%
\textbf{Pertes ennemies et retraites}%
}%
.
%
\end{description}%
Si vous perdez une région suite à l'invasion d'un adversaire, vous devez reprendre en main vos jetons Peuple et :
%
\begin{itemize}%
\item%
%
 Défausser définitivement 1 de ces jetons.
%
\item%
%
 À la fin du tour du joueur actif, poser les jetons encore en main sur les territoires que vous contrôlez encore. S'il ne vous reste plus de territoire, vous pourrez conquérir par les bords du plateau lors de votre prochain tour.
%
\end{itemize}%
\begin{description}%
\item[{-} ]%
%
 %
\textcolor{mygreen}{%
\textbf{Redéploiement}%
}%
\textit{ }%
 : Lorsque le joueur actif a terminé de jouer, il peut redéployer ses jetons Peuple sur les territoires encore en sa possession. Il doit juste garder un minimum de 1 jeton Peuple sur chaque territoire.
%
\end{description}

%
\subsection{ MARQUER DES POINTS DE VICTOIRE
}%
\label{subsec:MARQUERDESPOINTSDEVICTOIRE}%
\begin{description}%
\item[{-} ]%
%
 Recevez 1 jeton de victoire par région occupée.
%
\item[{-} ]%
%
 Vérifiez si les pouvoirs spéciaux de vos Peuples vous rapportent des points de victoire supplémentaires.
%
\item[{-} ]%
%
 Quand tous les joueurs ont marqué leurs points de victoire, avancez le marqueur de tour d'une case.
%
\end{description}

%
\subsection{ POURSUIVRE OU DÉCLINER
}%
\label{subsec:POURSUIVREOUDCLINER}%
\textcolor{mygreen}{%
\textbf{Poursuivre son extension avec son peuple actif}%
}%

%
\begin{description}%
\item[{-} ]%
%
 Au début de chaque tour suivant, reprenez en main tous les pions de votre Peuple actif déployés sur le plateau en
%
\end{description}%
laissant 1 pion dans chaque région occupée.
%
\begin{description}%
\item[{-} ]%
%
 Si vous voulez disposer de d'avantage de pions Peuple, vous pouvez %
\textcolor{mygreen}{%
\textbf{abandonner un territoire}%
}%
\textit{ }%
 en récupérant le seul pion qui s'y trouve. Si vous abandonnez tous vos territoires, vous devez appliquer la règle de Première conquête
%
\item[{-} ]%
%
 %
\textcolor{mygreen}{%
\textbf{Passer en Déclin :}%
}%

%
\item[{-} ]%
%
 Lorsque vous estimez que votre Peuple est trop diminué pour poursuivre ses conquêtes et vous rapporter le maximum de P.V, vous pouvez le faire passer en déclin.
%
\item[{-} ]%
%
 Retournez votre tuile Peuple coté inactif et défaussez son pouvoir spécial (sauf mention contraire).
%
\item[{-} ]%
%
 Gardez 1 seul pion Peuple sur chaque territoire et retournez{-}les tous côté Déclin. Les autres pions sont défaussés.
%
\item[{-} ]%
%
 Lorsqu'un Peuple disparaît totalement du plateau, le joueur concerné doit défausser sa tuile.
%
\item[{-} ]%
%
 Comptez vos points victoire et votre tour s’achève.
%
\item[{-} ]%
%
 Vous pourrez choisir une nouvelle combinaison Peuple/Action lors de votre prochain tour.
%
\item[{-} ]%
%
 Vous engagerez votre nouveau Peuple sur le plateau selon les règles de la Premiere conquête.
%
\item[{-} ]%
%
 À la fin de votre tour, vous compterez les P.V des territoires occupés par votre nouveau Peuple ET par votre Peuple en Déclin.
%
\end{description}

%
\subsection{ POUVOIR DES PEUPLES
}%
\label{subsec:POUVOIRDESPEUPLES}%
\begin{description}%
\item[{-} ]%
%
\textcolor{mygreen}{%
\textbf{Amazones}%
}%
\textit{ }%
 : Vous disposez de 4 pions Amazone supplémentaire pour les attaques uniquement. Après le redéploiement, retirez ces 4 pions du plateau.%
\includegraphics[width=2cm,height=2cm,keepaspectratio]{C:/Users/badcy/Documents/Unif/BAC 3/Q1/Projet individuel/2223_INFOB318_QuickBGR/code/my_app/static/img/tmp/6.jpg}%

%

%
\item[{-} ]%
%
\textcolor{mygreen}{%
\textbf{Elfes}%
}%
\textit{ }%
 : Lorsque vous perdez un territoire, vous ne défaussez pas un de vos pions Elfe repris en main.
%
\item[{-} ]%
%
\textcolor{mygreen}{%
\textbf{Géants}%
}%
\textit{ }%
 : Peuvent conquérir les territoires adjacents à une montagne (qu'ils contrôlent) en dépensant 1 jeton de moins, Minimum 1 pion.
%
\item[{-} ]%
%
\textcolor{mygreen}{%
\textbf{Humains}%
}%
\textit{ }%
 : Rapportent 1 P.v supplémentaire par champs contrôlés par des Humains.
%
\item[{-} ]%
%
\textcolor{mygreen}{%
\textbf{Mages}%
}%
\textit{ }%
 : Rapportent 1 P.V supplémentaire par territoire avec une source magique contrôlé par des Mages.
%
\item[{-} ]%
%
\textcolor{mygreen}{%
\textbf{Mi{-}portions}%
}%
\textit{ }%
 : Ils peuvent entrer sur le plateau par n'importe quelle région. Placez une Tanière dans les deux premières régions que vous prenez. Ces régions sont à présent imprenables et immunisées contre les pouvoirs des autres Peuples.
%
\item[{-} ]%
%
\textcolor{mygreen}{%
\textbf{Nains}%
}%
\textit{ }%
 : Rapportent 1 P.v supplémentaire par Mines contrôlées par des Nains (même en Déclin).
%
\item[{-} ]%
%
\textcolor{mygreen}{%
\textbf{Orcs}%
}%
\textit{ }%
 : Toute région non{-}vide conquise par vos Orcs durant ce tour rapporte 1 jeton de victoire supplémentaire en fin de tour.
%
\item[{-} ]%
%
\textcolor{mygreen}{%
\textbf{Sorciers}%
}%
\textit{ }%
 : Ils peuvent remplacer un pion actif de chaque adversaire par un des leurs (pris dans la réserve generale) pour conquérir une région. Le pion en question doit être le seul de son Peuple dans la région. Cette région doit être adjacente à une région appartenant aux Sorciers. Le pion adverse remplacé est remis dans la réserve générale (Elfe aussi).
%
\item[{-} ]%
%
\textcolor{mygreen}{%
\textbf{Squelettes}%
}%
\textit{ }%
 : Lors du Redéploiement de vos troupes, prenez 1 nouveau pion Squelette de la réserve générale pour toute série de 2 régions non{-}vides conquises par vos Squelettes lors de ce tour.
%
\item[{-} ]%
%
\textcolor{mygreen}{%
\textbf{Tritons}%
}%
\textit{ }%
 : Ils peuvent conquérir tout territoire adjacent aux mers ou au lac avec 1 pion de moins. Minimum 1 pion.
%
\item[{-} ]%
%
\textcolor{mygreen}{%
\textbf{Troll}%
}%
\textit{ }%
 : Placez un Antre de Troll dans chaque région occupée par vos Trolls. L’Antre de Troll augmente la défense de +1.
%
\item[{-} ]%
%
\textcolor{mygreen}{%
\textbf{Zombie}%
}%
\textit{ }%
 : Lorsque vos Zombies passent en déclin, tous leurs pions restent sur le plateau. Ils peuvent continuer à conquérir de nouvelles régions lors des prochains tours.%
\includegraphics[width=2cm,height=2cm,keepaspectratio]{C:/Users/badcy/Documents/Unif/BAC 3/Q1/Projet individuel/2223_INFOB318_QuickBGR/code/my_app/static/img/tmp/7.jpg}%

%

%
\end{description}

%
\subsection{ POUVOIRS SPÉCIAUX
}%
\label{subsec:POUVOIRSSPCIAUX}%
\begin{description}%
\item[{-} ]%
%
\textcolor{mygreen}{%
\textbf{Alchimistes}%
}%
\textit{ }%
 : Tant que votre Peuple n'est pas en déclin, prenez 2 jetons de victoire supplémentaires à la fin de chaque tour.
%
\item[{-} ]%
%
\textcolor{mygreen}{%
\textbf{Ancestraux}%
}%
\textit{ }%
 : Vous pouvez conserver 2 Peuples en Déclin.
%
\item[{-} ]%
%
\textcolor{mygreen}{%
\textbf{Armés}%
}%
\textit{ }%
 : Toute région peut être conquise avec 1 pion de Peuple de moins que nécessaire (minimum 1 pion).
%
\item[{-} ]%
%
\textcolor{mygreen}{%
\textbf{Aux deux Héros}%
}%
\textit{ }%
 : À la fin de chaque tour, déposez vos 2 héros sur 2 territoires que vous contrôlez. Ils sont immunisés aux Pouvoirs Spéciaux des Peuples adverses.
%
\item[{-} ]%
%
\textcolor{mygreen}{%
\textbf{Bâtisseurs}%
}%
\textit{ }%
 : Une fois par tour, placez 1 Forteresse dans une région que vous occupez. Elle augmente la défense du territoire de + 1 et permet de gagner 1 jeton P.V supplémentaire.
%
\item[{-} ]%
%
\textcolor{mygreen}{%
\textbf{Berserks}%
}%
\textit{ }%
 : Vous pouvez utiliser le Dé de renfort avant chaque conquête (minimum 1 pion par conquête).
%
\item[{-} ]%
%
\textcolor{mygreen}{%
\textbf{Des Cavernes}%
}%
\textit{ }%
 : Toute région qui comporte une Caverne peut être conquise avec un pion de Peuple de moins. Les territoires avec une Caverne sont reliés entre eux.
%
\item[{-} ]%
%
\textcolor{mygreen}{%
\textbf{Des Collines}%
}%
\textit{ }%
 : Prenez 1 jeton de victoire supplémentaire pour chaque Colline que vous occupez en fin de tour.
%
\item[{-} ]%
%
\textcolor{mygreen}{%
\textbf{Des Forêts}%
}%
\textit{ }%
 : Prenez 1 jeton de victoire supplémentaire pour chaque Forêt que vous occupez en fin de tour.
%
\item[{-} ]%
%
\textcolor{mygreen}{%
\textbf{Des Marais}%
}%
\textit{ }%
 : Prenez 1 jeton de victoire supplémentaire pour chaque Marais que vous occupez en fin de tour.
%
\item[{-} ]%
%
\textcolor{mygreen}{%
\textbf{Diplomates}%
}%
\textit{ }%
 : À la fin de votre tour, choisissez un joueur dont vous n'avez pas attaqué son Peuple actif. Il ne pourra pas attaquer votre Peuple actif au prochain tour.
%
\item[{-} ]%
%
\textcolor{mygreen}{%
\textbf{Durs à cuire}%
}%
\textit{ }%
 : Vous pouvez poursuivre votre expansion ET passer en déclin immédiatement après avoir pris vos P.V Et leur Dragon : Une fois par tour, vous pouvez conquérir n'importe quelle région en utilisant votre Dragon et un seul pion de Peuple. Ce territoire est immunisé aux pouvoirs des Peuples adverses tant que le dragon s'y trouve.
%
\item[{-} ]%
%
\textcolor{mygreen}{%
\textbf{Fortunés}%
}%
\textit{ }%
 : Prenez 7 jetons de victoire supplémentaires à la fin de votre premier tour (1 seule fois). 
%
\item[{-} ]%
%
\textcolor{mygreen}{%
\textbf{Marchands}%
}%
: Prenez 1 jeton de victoire supplémentaire pour chaque région que vous occupez en fin de tour.
%
\item[{-} ]%
%
\textcolor{mygreen}{%
\textbf{Marins}%
}%
\textit{ }%
 : Vous pouvez considérer les mers et le lac comme trois régions vides et les conquérir.
%
\item[{-} ]%
%
\textcolor{mygreen}{%
\textbf{Montés}%
}%
\textit{ }%
 : Toute Colline et tout Champ peut être conquis avec un pion de Peuple de moins.
%
\item[{-} ]%
%
\textcolor{mygreen}{%
\textbf{Pillards}%
}%
\textit{ }%
 : Toute région non{-}vide conquise durant ce tour rapporte 1 jeton de victoire supplémentaire en fin de tour.
%
\item[{-} ]%
%
\textcolor{mygreen}{%
\textbf{Scouts}%
}%
\textit{ }%
 : Placez les 5 jetons de Campement dans la ou les régions que vous occupez pendant la phase de Redéploiement. Ils augmentent la défense de la région de +1 point. Ils sont déplaçables à chaque tour.
%
\item[{-} ]%
%
\textcolor{mygreen}{%
\textbf{Volants}%
}%
\textit{ }%
 : Vous pouvez conquérir n'importe quelle région (sauf mer et lac), même si elle n'est pas adjacente à une des vôtres.
%
\end{description}

%
\sectionfont{\color{red}}%
\subsectionfont{\color{red}}%
\subsubsectionfont{\color{red}}%
\section{ Fin de partie
}%
\label{sec:Findepartie}%
\textcolor{red}{\rule{18cm}{0.07cm}}\break%
\begin{description}%
\item[{-} ]%
%
 %
\textcolor{red}{%
\textbf{Lorsque le jeton compte{-}tours atteint la dernière case de la piste compte{-}tours, c’est le dernier tour.}%
}%

%
\item[{-} ]%
%
 %
\textcolor{red}{%
\textbf{Chaque joueur joue alors une dernière fois.}%
}%

%
\item[{-} ]%
%
 %
\textcolor{red}{%
\textbf{Comptez vos jetons point de victoire. Le joueur possédant le plus de points de victoire, remporte la partie.}%
}%
\end{description}

%
\end{document}
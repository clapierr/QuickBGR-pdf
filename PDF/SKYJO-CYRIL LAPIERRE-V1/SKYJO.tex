\documentclass{article}%
\usepackage[T1]{fontenc}%
\usepackage[utf8]{inputenc}%
\usepackage{lmodern}%
\usepackage{textcomp}%
%
\title{SKYJO}%
\author{Cyril Lapierre}%
\date{\today}%
%
\begin{document}%
\pagestyle{empty}%
\normalsize%
\maketitle%
\section{ Mise en place
}%
\label{sec:Miseenplace}%
\begin{itemize}%
\item%
%
 Chaque joueur reçoit 12 cartes face cachée.
%
\item%
%
 Avec vos 12 cartes toujours face cachée, constituez un tableau de 4 colonnes de 3 cartes.
%
\item%
%
 Retournez face visible, 2 des 12 cartes de votre tableau.
%
\item%
%
 Formez une pioche avec les cartes restantes.
%
\item%
%
 Révélez la première carte de la pioche, elle constitue le début de la pile de défausse.
%
\item%
%
 Le joueur, dont la somme des 2 cartes visibles est la plus élevée, commence la partie
%
\end{itemize}

%
\section{ Tour de jeu
}%
\label{sec:Tourdejeu}%
LE JEU SE DÉROULE EN PLUSIEURS MANCHES SUCCESSIVES. À CHAQUE TOUR UN JOUEUR DOIT CHOISIR UNE CARTE ENTRE LA DÉFAUSSE OU LA PIOCHE.


%
\subsection{ PRENDRE LA PREMIÈRE CARTE RETOURNÉE DE LA DÉFAUSSE
}%
\label{subsec:PRENDRELAPREMIRECARTERETOURNEDELADFAUSSE}%
Vous devez immédiatement échanger cette carte avec l’une de vos 12 cartes (visible ou cachée) :
%
\begin{itemize}%
\item%
%
 La nouvelle carte est posée face visible.
%
\item%
%
 L'ancienne carte est déposée sur le dessus de la pile de défausse, face visible.
%
\end{itemize}%
Vous ne pouvez pas regarder vos cartes cachées avant de faire un échange.


%
\subsection{ PRENDRE LA PREMIÈRE CARTE FACE CACHÉE DE LA PIOCHE
}%
\label{subsec:PRENDRELAPREMIRECARTEFACECACHEDELAPIOCHE}%
Une fois la carte piochée, vous pouvez au choix :
%
\begin{itemize}%
\item%
%
 L'échanger contre l'une de vos 12 cartes (visible ou cachée).
%
\item%
%
 La déposer sur le dessus de la pile de défausse car elle ne vous intéresse pas. Dans ce cas, vous devez retourner
%
\end{itemize}%
face visible une de vos cartes cachées.


%
\subsection{ RÈGLE SPÉCIALE
}%
\label{subsec:RGLESPCIALE}%
Si suite à un échange ou après avoir révélé une carte face cachée, vous possédez 3 cartes identiques dans une
%
colonne, retirez ces 3 cartes de votre tableau et placez{-}les sur la défausse


%
\section{ Fin de la manche
}%
\label{sec:Findelamanche}%
Le premier joueur à révéler toutes ses cartes (toutes les cartes sont face visible), met fin à la manche.
%
\begin{itemize}%
\item%
%
 Finissez le tour en cours.
%
\item%
%
 Les joueurs ayant encore des cartes face cachée les retournent face visible. Appliquez la Règle spéciale des 3 cartes identiques si le cas se présente.
%
\item%
%
 Chaque joueur comptabilise son total de points en ajoutant toutes les valeurs des cartes visibles dans son tableau :
%
\end{itemize}%
{-}Si le total de points est positif, vous devez le rajouter à votre total de points des manches précédentes.
%
\begin{itemize}%
\item%
%
 Si le total est négatif, vous devez le soustraire à votre total de points des manches précédentes.
%
\end{itemize}%
{-}Si le joueur qui a terminé la manche, n’a pas strictement obtenu le plus petit nombre de points de cette manche,alors, la somme de ses points pour cette manche est doublée.


%
\section{ Fin de la partie
}%
\label{sec:Findelapartie}%
\textbf{{-} Le jeu se termine dès qu’un joueur atteint 100 points ou plus.}%

%
\textbf{{-} Le joueur avec le plus petit score gagne la partie.}

%
\end{document}
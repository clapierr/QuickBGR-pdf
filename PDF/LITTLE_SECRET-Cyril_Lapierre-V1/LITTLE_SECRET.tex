\documentclass{article}%
\usepackage[T1]{fontenc}%
\usepackage[utf8]{inputenc}%
\usepackage{lmodern}%
\usepackage{textcomp}%
%
\title{LITTLE\_SECRET}%
\author{Cyril\_Lapierre}%
\date{\today}%
%
\begin{document}%
\pagestyle{empty}%
\normalsize%
\maketitle%
\section{Game Rules}%
\label{sec:GameRules}%
\# Set up
\newline%
{-} Préparez l'équivalent de **4/5 Missions**. Chaque Mission est composée d'une carte **Codes secrets** et d'un set de cartes **Discipl**e,** Infiltrés** et Journaliste reparties selon le nombre de joueurs.
\newline%
{-} Choisissez une Mission et distribuez une carte à chaque joueur. Posez au **centre** de la table la carte **Codes secrets**.
\newline%
{-} Pour chaque mission, les mots des cartes **Disciples** ont la même couleur que ceux de la carte Codes Secrets alors que les cartes Infiltrés ont une couleur **différente**.
\newline%
{-} **Le premier joueur** choisit un chiffre entre 1 et 10 pour sélectionner le **Mot de passe** sur sa carte.
\newline%
 \# Round of play
\newline%
 **Chacun leur tour, les joueurs annoncent un mot en rapport avec le mot de passe sélectionné. L'infiltré À un mot de passe légèrement différent des autres et le journaliste, qui ne connait pas le mot de passe, doit improviser en s'inspirant des mots précédemment annoncés**.
\newline%
 \#\# Donner un indice
\newline%
{-} Chacun leur tour, les joueurs annonce un Indice (un mot) en rapport avec le Mot de passe préalablement sélectionné
\newline%
indiqué sur leur carte.
\newline%
{-} À part le Journaliste (qui connait son rôle), les autres joueurs ne savent pas s'ils sont Disciples ou Infiltrés.
\newline%
{-} Les Disciples : Ils reçoivent tous le même Mot de passe. Leur objectif est de démasquer les Infiltrés et le
\newline%
Journaliste.
\newline%
{-} Les InfiltrÉs : Ils reçoivent un Mot de passe légèrement différent de celui des Disciples. Leur objectif est
\newline%
de se faire passer pour des Disciples.
\newline%
{-} Le Journaliste : Il ne reçoit aucun Mot de passe. Son but est de faire croire qu’il en a un, tout en essayant
\newline%
de deviner celui des Disciples.
\newline%
 \#\# Débattre
\newline%
  {-} Une fois que tous les joueurs ont donné leur Indice, débattez entre vous pour déterminer qui peut
\newline%
être le **Journaliste** ou un **Infiltré**.
\newline%
{-} Votez tous en même temps en pointant de votre doigt la personne que vous soupçonnez.
\newline%
{-} Le joueur désigné par la majorité des votants est ** éliminé** :
\newline%
• Si c’était le Journaliste, il peut encore tenter de deviner le Mot de passe des Disciples. Il l'annonce
\newline%
à haute voix puis vérifie sur la carte Codes secrets. S’il réussit la partie s’arrête et il marque 3 points.
\newline%
S’il échoue, la partie continue.
\newline%
• Si ce n’était pas le Journaliste, il regarde discrètement la carte Codes secrets et annonce aux autres joueurs
\newline%
s’il était un Disciple (il a le même mot que sur la carte Codes secrets) ou un Infiltré (il a un mot différent).
\newline%
 \# End of game
\newline%
 **Le premier joueur atteignant 5 points est déclaré vainqueur**

%
\end{document}
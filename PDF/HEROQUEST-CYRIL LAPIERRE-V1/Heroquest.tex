\documentclass{scrartcl}%
\usepackage[T1]{fontenc}%
\usepackage[utf8]{inputenc}%
\usepackage{lmodern}%
\usepackage{textcomp}%
\usepackage{lastpage}%
\usepackage{geometry}%
\geometry{margin=0.7in}%
\usepackage{xcolor}%
\usepackage{fancyhdr}%
%
\usepackage{sectsty}%
\usepackage{graphicx}%
\usepackage{xcolor}%
\definecolor{mygreen}{RGB}{58,170,53}%
\title{\includegraphics[width=8cm,height=4cm,keepaspectratio]{C:/Users/badcy/Documents/Unif/BAC 3/Q1/Projet individuel/2223_INFOB318_QuickBGR/code/my_app/static/img/tmp/Box-Heroquest.jpg}\break Heroquest }%
\author{Cyril Lapierre}%
\date{\today \break Tags: Adventure, Exploration, Fantasy, Fighting, Miniatures}%
\fancypagestyle{header}{%
\renewcommand{\headrulewidth}{1pt}%
\renewcommand{\footrulewidth}{1pt}%
\fancyhead{%
}%
\fancyfoot{%
}%
\fancyhead[L]{%
Page \thepage%
}%
\fancyhead[C]{%
\includegraphics[width=4cm,height=1cm,keepaspectratio]{C:/Users/badcy/Documents/Unif/BAC 3/Q1/Projet individuel/2223_INFOB318_QuickBGR/code/my_app/static/img/logo/QuickBGR-black-green.png}%
}%
\fancyhead[R]{%
\today%
}%
\fancyfoot[C]{%
\includegraphics[width=4cm,height=1cm,keepaspectratio]{C:/Users/badcy/Documents/Unif/BAC 3/Q1/Projet individuel/2223_INFOB318_QuickBGR/code/my_app/static/img/logo/QuickBGR-black-green.png}%
}%
}%
%
\begin{document}%
\normalsize%
\maketitle\thispagestyle{header}%
\pagestyle{header}%
\sectionfont{\color{blue}}%
\subsectionfont{\color{blue}}%
\subsubsectionfont{\color{blue}}%
\section{ Mise en place
}%
\label{sec:Miseenplace}%
\textcolor{blue}{\rule{18cm}{0.07cm}}\break%
%
\begin{center}\includegraphics[width=4cm,height=4cm,keepaspectratio]{C:/Users/badcy/Documents/Unif/BAC 3/Q1/Projet individuel/2223_INFOB318_QuickBGR/code/my_app/static/img/tmp/1.jpg}\end{center}%

%
\begin{description}%
\item[{-} ]%
%
 Chaque joueur choisit un rôle. Au moins un des participants doit jouer le rôle de%
\textcolor{blue}{\textbf{\textit{ Zargon}}}%
.%
\textcolor{blue}{%
\textbf{ À moins de 5 joueurs}%
}%
, vous pouvez incarnez %
\textcolor{blue}{%
\textbf{plusieurs}%
}%
\textit{ }%
 Héros (Barbare, Nain, Elfe, Enchanteur).
%
\item[{-} ]%
%
 Zargon récupère le Livre des quêtes. La lecture de la Quête se déroule ainsi :
%
\end{description}%
\begin{enumerate}%
\item%
%
 %
\textcolor{blue}{%
\textbf{Le chapitre de la Quête}%
}%
\textit{ }%
 (Quête 1 {-} L'épreuve), %
\textcolor{blue}{%
\textbf{à lire silencieusement}%
}%
.
%
\item%
%
 Le texte sur%
\textcolor{blue}{%
\textbf{ Parchemin}%
}%
\textit{ }%
 (défi que les Héros rencontreront et la récompense obtenue en cas de succès) à lire à voix haute à l'ensemble des joueurs.
%
\item%
%
 La carte de Quête. Seul %
\textcolor{blue}{\textbf{\textit{Zargon}}}%
\textit{ }%
 en prend connaissance. Elle indique où apparaîtront les Monstres et le mobilier en cours de partie. Ne rien disposer sur le plateau pour l'instant.
%
\item%
%
 Le %
\textcolor{blue}{%
\textbf{Livre des quêtes}%
}%
. Il décrit ce qui arrivera dans certaines Pièces et les situations spéciales que les Héros devront affronter. À lire entièrement et %
\textcolor{blue}{%
\textbf{silencieusement}%
}%
. Zargon révélera ces informations au fur et à mesure que la Quête progressera.
%
\end{enumerate}%
\begin{description}%
\item[{-} ]%
%
 Placer le plateau de jeu au centre de la table (Le logo HeroQuest à la droite de Zargon).
%
\item[{-} ]%
%
 Placer les %
\textcolor{blue}{%
\textbf{4 cartes de personnage}%
}%
\textit{ }%
 face visible. Vous y trouverez des capacités qui évolueront au cours de la partie :
%
\end{description}%
\begin{itemize}%
\item%
%
 %
\textcolor{blue}{%
\textbf{ Dés d’attaque}%
}%
\textit{ }%
 : Il s'agit de la capacité d'attaque de l'Arme du Héros.
%
\item%
%
 %
\textcolor{blue}{%
\textbf{Dés de défense}%
}%
\textit{ }%
 : Il s'agit de la capacité du Héros à éviter ou à encaisser les dégâts de l'ennemi.
%
\item%
%
 %
\textcolor{blue}{%
\textbf{Points de vie}%
}%
\textit{ }%
 : Ils représentent la force physique du Héros.
%
\item%
%
 %
\textcolor{blue}{%
\textbf{Points d'esprit}%
}%
\textit{ }%
 : Ils représentent l'intelligence, la sagesse et la résistance à la magie du Héros.
%
\end{itemize}%
\begin{description}%
\item[{-} ]%
%
 %
\textcolor{blue}{%
\textbf{Chaque joueur}%
}%
\textit{ }%
 reçoit sa figurine, la carte de son personnage ainsi qu'un feuillet à remplir avec les éléments indiqués sur la carte.
%
\item[{-} ]%
%
 %
\textcolor{blue}{\textbf{\textit{ Zargon}}}%
\textit{ }%
 récupère un paravent pour dissimuler le Livre des Quêtes posé devant lui.
%
\item[{-} ]%
%
 Formez une réserve avec les éléments restants du jeu (Portes, le mobilier, les Monstres et toutes les tuiles cartonnées). Regardez la carte de la Quête dans le Livre des quêtes. Ne placez sur le plateau que les éléments se trouvant dans la pièce de départ.
%
\item[{-} ]%
%
 Séparez les cartes en 9 piles distinctes :
%
\end{description}%
\begin{itemize}%
\item%
%
 %
\textcolor{blue}{%
\textbf{Équipements}%
}%
,
%
\item%
%
 %
\textcolor{blue}{%
\textbf{Trésors}%
}%
\textit{ }%
 (Mélanger ces cartes avant chaque Quête.),
%
\item%
%
 %
\textcolor{blue}{%
\textbf{Artefacts}%
}%
\textit{ }%
 (Gardez ces cartes derrière l'écran du maître de jeu).
%
\item%
%
 %
\textcolor{blue}{%
\textbf{Monstres}%
}%
\textit{ }%
 (Étalez face visible les huit cartes près du plateau),
%
\item%
%
 %
\textcolor{blue}{%
\textbf{Sorts de la Terreur}%
}%
\textit{ }%
 : Zargon doit garder ces cartes derrière l'écran du maître de jeu.
%
\item%
%
 %
\textcolor{blue}{%
\textbf{Sorts X4}%
}%
\textit{ }%
 (air, feu, eau, terre) : Les groupes de Sorts sont répartis entre l'Enchanteur et l'Elfe.
%
\end{itemize}%
\begin{description}%
\item[{-} ]%
%
 Placez %
\textcolor{blue}{%
\textbf{3}%
}%
\textit{ }%
 Dés de %
\textcolor{blue}{%
\textbf{combat}%
}%
\textit{ }%
 blancs et les 2 dés rouges près du plateau pour les Héros. Gardez trois Dés de combat pour Zargon.
%
\end{description}%
%
\begin{center}\includegraphics[width=4cm,height=4cm,keepaspectratio]{C:/Users/badcy/Documents/Unif/BAC 3/Q1/Projet individuel/2223_INFOB318_QuickBGR/code/my_app/static/img/tmp/2.jpg}\end{center}%



%
\sectionfont{\color{mygreen}}%
\subsectionfont{\color{mygreen}}%
\subsubsectionfont{\color{mygreen}}%
\section{ Tour de jeu
}%
\label{sec:Tourdejeu}%
\textcolor{mygreen}{\rule{18cm}{0.07cm}}\break%
\textcolor{mygreen}{%
\textbf{Zargon lit aux autres aventuriers le texte inscrit sur le parchemin de la quÊte concernÉe. Une partie est composÉE de plusieurs manches durant lesquelles les hÉros accomplissent une action chacun avant que Zargon ne dÉplace les monstres et active leurs pouvoirs.}%
}%

%
%
\begin{center}\includegraphics[width=4cm,height=4cm,keepaspectratio]{C:/Users/badcy/Documents/Unif/BAC 3/Q1/Projet individuel/2223_INFOB318_QuickBGR/code/my_app/static/img/tmp/3.jpg}\end{center}%



%
\subsection{ Tour des Héros
}%
\label{subsec:TourdesHros}%
Un Héros peut se déplacer puis effectuer une action ou le %
\textcolor{mygreen}{%
\textbf{contraire}%
}%
. 


%
\subsubsection{ Déplacements d'un Héros :
}%
\label{ssubsec:DplacementsdunHros}%
\begin{description}%
\item[{-} ]%
%
 Pour déterminer le nombre de cases d'un déplacement, le Héros lance les %
\textcolor{mygreen}{%
\textbf{deux dés rouges}%
}%
.
%
\end{description}%
\begin{itemize}%
\item%
%
 Il n'est pas nécessaire de parcourir entièrement la distance indiquée par les dés.
%
\item%
%
 Un Héros ne peut pas passer par{-}dessus les %
\textcolor{mygreen}{%
\textbf{Monstres}%
}%
, traverser les Murs ou se déplacer en diagonale.
%
\item%
%
 Vous pouvez déplacer un Héros %
\textcolor{mygreen}{%
\textbf{par{-}dessus}%
}%
\textit{ }%
 un autre Héros..
%
\item%
%
 Vous devez emprunter une Porte pour entrer dans une pièce.
%
\item%
%
 Vous ne pouvez pas partager une case avec un autre Héros (sauf case de l'Escalier et Oubliettes).
%
\end{itemize}%
\begin{description}%
\item[{-} ]%
%
 Les Héros qui se déplacent peuvent regarder dans les Corridors et les Pièces dont la Porte est ouverte. Au début, toutes les Portes sont fermées. Une fois ouverte, une Porte ne peut pas être refermée.
%
\end{description}%
\begin{itemize}%
\item%
%
 %
\textcolor{mygreen}{%
\textbf{Regarde}%
}%
r ne compte pas comme une action.
%
\item%
%
 Vous pouvez demander à %
\textcolor{mygreen}{\textbf{\textit{Zargon}}}%
\textit{ }%
 d'ouvrir une Porte adjacente à votre case. Cela ne compte pas comme une action.
%
\end{itemize}%
\begin{description}%
\item[{-} ]%
%
 %
\textcolor{mygreen}{\textbf{\textit{Zargon}}}%
\textit{ }%
 doit suivre les déplacements des Héros attentivement en se référant constamment à la carte de la Quête et au Livre des Quêtes :
%
\end{description}%
\begin{itemize}%
\item%
%
 Quand un Héros regarde dans un lieu, %
\textcolor{mygreen}{\textbf{\textit{Zargon}}}%
\textit{ }%
 doit placer tous les éléments illustrés sur sa carte, que le Héros peut voir dans sa ligne de vision (non dissimulé par un Mur, Porte fermée....). Cela peut être des Portes fermées, %
\textcolor{mygreen}{%
\textbf{des Coffres à Trésors}%
}%
, des %
\textcolor{mygreen}{%
\textbf{Monstres}%
}%
mais pas les%
\textcolor{mygreen}{%
\textbf{ Pièges}%
}%
\textit{ }%
 ni la nature des Trésors qui restent dissimulés. Les cases obstruées visibles par les Héros doivent être posées sur le plateau. %
\textcolor{mygreen}{%
\textbf{Aucun Héros ni Monstres}%
}%
\textit{ }%
 ne peuvent %
\textcolor{mygreen}{%
\textbf{traverser ces cases}%
}%
.
%
\end{itemize}

%
\subsubsection{ Action des Héros :
}%
\label{ssubsec:ActiondesHros}%
À leur tour, les Héros peuvent réaliser 1 action parmi les suivantes :
%
\textcolor{mygreen}{%
\textbf{Attaquer :}%
}%

%
\begin{description}%
\item[{-} ]%
%
 Un Héros peut attaquer n’importe quel Monstre se trouvant sur une case adjacente à la sienne.
%
\item[{-} ]%
%
 %
\textcolor{mygreen}{%
\textbf{Une}%
}%
\textit{ }%
 seule attaque par tour est possible.
%
\item[{-} ]%
%
 Lancez les Dés de combat %
\textcolor{mygreen}{%
\textbf{blanc}%
}%
\textit{ }%
 selon la capacité d'attaque de votre Arme. Vous pourrez acquérir de meilleures%
\textcolor{mygreen}{%
\textbf{ Armes}%
}%
\textit{ }%
 et %
\textcolor{mygreen}{%
\textbf{Armures}%
}%
\textit{ }%
 en dépensant l'or récolté à %
\textcolor{mygreen}{%
\textbf{Armurerie}%
}%
\textit{ }%
 entre chaque Quête.%
\begin{center}\includegraphics[width=3cm,height=3cm,keepaspectratio]{C:/Users/badcy/Documents/Unif/BAC 3/Q1/Projet individuel/2223_INFOB318_QuickBGR/code/my_app/static/img/tmp/4.jpg}\end{center}%

%
\end{description}%
%
\includegraphics[width=2cm,height=2cm,keepaspectratio]{C:/Users/badcy/Documents/Unif/BAC 3/Q1/Projet individuel/2223_INFOB318_QuickBGR/code/my_app/static/img/tmp/5.jpg}%
{-} %
\textcolor{mygreen}{%
\textbf{Chaque Crâne}%
}%
\textit{ }%
 obtenu est considéré comme un coup réussi et représente %
\textcolor{mygreen}{%
\textbf{1 point de vie}%
}%
\textit{ }%
 pouvant être enlevé au Monstre. Si vous n’obtenez aucun Crâne, l'attaque est ratée. Si les points de vie du Monstre sont réduits à zéro,%
\textcolor{mygreen}{%
\textbf{ il est mort et retiré du jeu}%
}%
.
%
\begin{description}%
\item[{-} ]%
%
 %
\textcolor{mygreen}{%
\textbf{Le Monstre peut}%
}%
\textit{ }%
 se défendre après l’attaque du Héros en lançant le nombre de Dés de défense indiqué dans le tableau des Monstres sur l'écran du maître de jeu. Chaque Bouclier noir obtenu annule un coup porté par le Héros qui attaque.
%
\item[{-} ]%
%
 Si le Monstre%
\textcolor{mygreen}{%
\textbf{ n'est pas mort}%
}%
, %
\textcolor{mygreen}{%
\textbf{Zargon}%
}%
inscrit les dégâts en plaçant une tuile avec un Crâne sous la figurine du Monstre.
%
\item[{-} ]%
%
 Si le Monstre %
\textcolor{mygreen}{%
\textbf{survit à l'attaque}%
}%
, il ne peut pas attaquer le Héros avant le tour suivant de%
\textcolor{mygreen}{%
\textbf{ Zargon}%
}%
. 
%
\end{description}%
\textcolor{mygreen}{%
\textbf{Lancer un Sort :}%
}%

%
\begin{description}%
\item[{-} ]%
%
 Durant leur tour, %
\textcolor{mygreen}{%
\textbf{l'Elfe}%
}%
\textit{ }%
 et %
\textcolor{mygreen}{%
\textbf{l'Enchanteur}%
}%
\textit{ }%
 peuvent lancer des Sorts sur une cible visible (une ligne ininterrompue peut être tracée entre le lanceur de Sort et sa cible.)
%
\item[{-} ]%
%
 Un Héros peut lancer %
\textcolor{mygreen}{%
\textbf{un Sort sur lui{-}même}%
}%
, sur un autre Héros ou sur un Monstre.
%
\item[{-} ]%
%
 Une fois le %
\textcolor{mygreen}{%
\textbf{Sort lancé}%
}%
, sa carte est %
\textcolor{mygreen}{%
\textbf{retirée du jeu}%
}%
.
%
\end{description}%
\textcolor{mygreen}{%
\textbf{ Chercher des Trésors :}%
}%

%
\begin{description}%
\item[{-} ]%
%
 Un Héros peut chercher des Trésors seulement s'il %
\textcolor{mygreen}{%
\textbf{n'y a pas de Monstres dans la pièce}%
}%
.
%
\item[{-} ]%
%
 Le Héros %
\textcolor{mygreen}{%
\textbf{annonce à haute voi}%
}%
x qu'il désire chercher un Trésor dans la pièce où il se trouve. Il y a deux types de Trésors :
%
\end{description}%
\begin{itemize}%
\item%
%
 %
\textcolor{mygreen}{%
\textbf{Trésor spécial}%
}%
\textit{ }%
 (indiqué dans les notes de Quêtes) : Zargon doit lire sa description à voix haute.%
\textcolor{mygreen}{%
\textbf{ Le Trésor ne peut être découvert qu'une seule fois}%
}%
.
%
\item%
%
 %
\textcolor{mygreen}{%
\textbf{Trésor classique}%
}%
\textit{ }%
 : S'il n'y a aucun Trésor spécial dans la pièce, le Héros concerné pioche une carte Trésor. Sur ces cartes le Héros peut trouver des %
\textcolor{mygreen}{%
\textbf{Potions}%
}%
\textit{ }%
 ou de l'Or ou %
\textcolor{mygreen}{%
\textbf{des Monstres}%
}%
\textit{ }%
 et des Pièges :
%
\end{itemize}%
\begin{enumerate}%
\item%
%
 %
\textcolor{mygreen}{%
\textbf{Monstres}%
}%
\textit{ }%
 : Dans ce cas, Zargon doit placer un Monstre à côté du chercheur de Trésor et lancer immédiatement les Dés d'attaque. Le %
\textcolor{mygreen}{%
\textbf{Héros doit se défendre}%
}%
\textit{ }%
 contre l'attaque du Monstre errant en lançant les Dés de combat. Il peut ensuite continuer son tour.
%
\item%
%
 %
\textcolor{mygreen}{%
\textbf{Pièges}%
}%
\textit{ }%
 : Le Héros ayant pioché la carte avec un Piège doit la lire à voix haute et suivre les instructions. 
%
\end{enumerate}%
\textcolor{mygreen}{%
\textbf{Chercher des Portes secrètes :}%
}%

%
\begin{description}%
\item[{-} ]%
%
 Les Héros ne peuvent pas découvrir une Porte cachée à moins d'en faire %
\textcolor{mygreen}{%
\textbf{expressément la demande}%
}%
.
%
\item[{-} ]%
%
 Si la demande est faite, Zargon doit alors révéler toutes les Portes secrètes situées dans le lieu où se trouve le Héros, en plaçant une tuile de Porte secrète sur la case correspondante. Ces Portes sont considérées %
\textcolor{mygreen}{%
\textbf{comme fermées}%
}%
.
%
\item[{-} ]%
%
 Les Héros ne peuvent pas chercher des Portes secrètes si un Monstre se trouve dans leur champ de vision
%
\end{description}%
\textcolor{mygreen}{%
\textbf{Chercher les PiÈges}%
}%
:
%
\begin{description}%
\item[{-} ]%
%
 %
\textcolor{mygreen}{%
\textbf{Seul Zargon}%
}%
\textit{ }%
 sait où se cachent les Pièges. Si un Héros passe sur une case avec un Piège,%
\textcolor{mygreen}{%
\textbf{ il le déclenche automatiquement.}%
}%

%
\item[{-} ]%
%
 Les Héros ne peuvent pas découvrir un Piège à moins d'en faire expressément la demande. Zargon doit alors montrer quelles cases sont piégées,%
\textcolor{mygreen}{%
\textbf{ sans les marquer d'une tuile Piège}%
}%
, %
\textcolor{mygreen}{%
\textbf{car ils sont pour l'instant cachés et non déclenchés}%
}%
.
%
\item[{-} ]%
%
 Les Héros ne peuvent pas chercher des Pièges si un Monstre se trouve dans leur champ de vision.
%
\item[{-} ]%
%
 Si un Héros trouve un Piège sur son chemin, il peut essayer de %
\textcolor{mygreen}{%
\textbf{sauter par{-}dessus.}%
}%
\textit{ }%
 :
%
\end{description}%
\begin{itemize}%
\item%
%
 Il doit rester au moins%
\textcolor{mygreen}{%
\textbf{ 2 cases à son déplacement }%
}%
pour atterrir sur une case libre adjacente.
%
\item%
%
 Le Héros doit lancer un Dé de combat. %
\textcolor{mygreen}{%
\textbf{S'il obtient un Crâne}%
}%
, il tombe dans le Piège et %
\textcolor{mygreen}{%
\textbf{perd des points de vie}%
}%
. %
\textcolor{mygreen}{%
\textbf{Zargon}%
}%
\textit{ }%
 doit alors placer la tuile de Piège correspondante sur la case. S'il n'obtient pas de Crâne, il réussit son saut.
%
\end{itemize}%
\begin{description}%
\item[{-} ]%
%
 %
\textcolor{mygreen}{%
\textbf{Les Monstres ne déclenchent pas les Pièges}%
}%
.
%
\item[{-} ]%
%
 Il existe 4 types de Pièges :
%
\end{description}%
\begin{itemize}%
\item%
%
 %
\textcolor{mygreen}{%
\textbf{Les Oubliettes}%
}%
\textit{ }%
 : Si un Héros passe sur une case avec une oubliette, Zargon l'arrête et place une tuile d'oubliette sur cette case. Le Héros perd alors %
\textcolor{mygreen}{%
\textbf{1 point de vie}%
}%
\textit{ }%
 dans sa chute en inscrivant cette perte sur sa feuille de personnage. C'est la fin de son tour. Un Héros ou un Monstre tombé dans les Oubliettes peut attaquer et se défendre mais avec un Dé de combat en moins. Normalement, un Héros peut sortir d'une oubliette%
\textcolor{mygreen}{%
\textbf{ à son prochain tour}%
}%
.
%
\item%
%
 %
\textcolor{mygreen}{%
\textbf{Les Pièges à éboulement}%
}%
\textit{ }%
 :
%
\end{itemize}%
Si un Héros passe sur une case avec un Piège à éboulement, Zargon place alors une tuile de Piège à éboulement sur cette case. Le Héros doit lancer %
\textcolor{mygreen}{%
\textbf{3 Dés de combat}%
}%
. Pour chaque Crâne obtenu, il perd 1 point de vie. Il ne peut pas lancer de dé pour se défendre. Il doit ensuite soit avancer, soit reculer vers une case libre et mettre fin à son tour. La %
\textcolor{mygreen}{%
\textbf{case piégée}%
}%
\textit{ }%
 constitue maintenant un obstacle infranchissable et peut isoler le Héros du reste du groupe selon son choix.
%
\begin{itemize}%
\item%
%
 %
\textcolor{mygreen}{%
\textbf{Piège à lance}%
}%
: Quand un Héros passe sur la case d'un Piège à lance, il doit lancer un Dé de combat. S'il obtient un Crâne, il perd 1 point de vie et %
\textcolor{mygreen}{%
\textbf{son tour est terminé}%
}%
. S'il obtient %
\textcolor{mygreen}{%
\textbf{un Bouclier noir ou blanc}%
}%
, il réussit à éviter la Lance et peut continuer son déplacement.%
\textcolor{mygreen}{%
\textbf{ Une fois déclenché, le Piège à lance n'est plus actif}%
}%
.
%
\item%
%
 %
\textcolor{mygreen}{%
\textbf{Coffres/meubles piégés}%
}%
: Si un Héros cherche des Trésors dans une pièce sans y avoir préalablement cherché des Pièges, il déclenche tout %
\textcolor{mygreen}{%
\textbf{Coffre/ Meuble piégé}%
}%
\textit{ }%
 se trouvant dans la pièce et son tour est terminé. Il %
\textcolor{mygreen}{%
\textbf{subit les effets du Piège }%
}%
(cf notes de Quête). Il est possible %
\textcolor{mygreen}{%
\textbf{de désarmer le Piège}%
}%
\textit{ }%
 puis ensuite chercher des Trésors dans le Coffre/Meuble désarmé pendant votre prochain tour.
%
\item%
%
\textit{Désarmer un PiÈge}%
\textit{:
}%
\end{itemize}%
\begin{description}%
\item[{-} ]%
%
 Le Héros doit d'abord %
\textcolor{mygreen}{%
\textbf{connaître son emplacement}%
}%
\textit{ }%
 et %
\textcolor{mygreen}{%
\textbf{posséder une Trousse à outils (ou être le nain)}%
}%
.
%
\item[{-} ]%
%
 Il se déplace sur la case piégée et lance %
\textcolor{mygreen}{%
\textbf{1 Dé de combat}%
}%
. S'il obtient un Crâne, il déclenche le Piège et perd des points de vie. S'il obtient un Bouclier blanc ou noir, il désarme le Piège.
%
\item[{-} ]%
%
 %
\textcolor{mygreen}{%
\textbf{Le Nain n'a pas besoin de Trousse}%
}%
. Avant de se déplacer, il doit se déplacer sur la case du Piège et lance 1 Dé de combat. S'il obtient un Bouclier noir, il déclenche le Piège et perd des points de vie. S'il obtient autre chose qu'un Bouclier noir, %
\textcolor{mygreen}{%
\textbf{il désarme le Piège}%
}%
. %
\textcolor{mygreen}{%
\textbf{Une oubliette désarmée se comporte comme une case normale}%
}%
.
%
\end{description}

%
\subsection{ Tour de Zargon
}%
\label{subsec:TourdeZargon}%
\begin{description}%
\item[{-} ]%
%
 Une fois que tous les joueurs ont terminé leur tour, c'est au tour de %
\textcolor{mygreen}{%
\textbf{Zargon}%
}%
. Il peut alors%
\textcolor{mygreen}{%
\textbf{ déplacer tous les Monstres}%
}%
\textit{ }%
 actuellement sur le plateau de jeu.
%
\item[{-} ]%
%
 Les Monstres ne peuvent pas%
\textcolor{mygreen}{%
\textbf{attaquer ou se déplacer}%
}%
\textit{ }%
 en %
\textcolor{mygreen}{%
\textbf{diagonale}%
}%
, passer à travers un Héros et partager leur case.
%
\item[{-} ]%
%
 Les Monstres ne se déplacent pas suite à un lancer de dé. Ils ont une capacité de déplacement %
\textcolor{mygreen}{%
\textbf{maximale indiquée sur leur carte}%
}%
.
%
\item[{-} ]%
%
 Chaque Monstre peut accomplir une des deux actions suivantes :
%
\end{description}%
\begin{itemize}%
\item%
%
 %
\textcolor{mygreen}{%
\textbf{Attaquer}%
}%

%
\end{itemize}%
%
\begin{center}\includegraphics[width=3cm,height=3cm,keepaspectratio]{C:/Users/badcy/Documents/Unif/BAC 3/Q1/Projet individuel/2223_INFOB318_QuickBGR/code/my_app/static/img/tmp/6.jpg}\end{center}%

%
\begin{description}%
\item[{-} ]%
%
 Un Monstre qui attaque un adversaire adjacent, doit lancer le nombre de %
\textcolor{mygreen}{%
\textbf{Dés d'attaque}%
}%
\textit{ }%
 indiqué dans le tableau des Monstres (écran du maître de jeu). Si aucun %
\textcolor{mygreen}{%
\textbf{Crâne}%
}%
\textit{ }%
 n'est obtenu, l'attaque est ratée.
%
\item[{-} ]%
%
 Chaque Crâne obtenu est considéré comme un coup réussi et représente %
\textcolor{mygreen}{%
\textbf{1 point de vie}%
}%
\textit{ }%
 pouvant être enlevé au Héros. Ce dernier doit immédiatement se défendre en lançant deux Dés de combat. Chaque %
\textcolor{mygreen}{%
\textbf{Bouclier blanc}%
}%
\textit{ }%
 obtenu bloque %
\textcolor{mygreen}{%
\textbf{1 coup du Monstre}%
}%
\textit{ }%
 qui attaque.
%
\item[{-} ]%
%
 Si les points de vie du Héros sont réduits à zéro, il est considéré comme %
\textcolor{mygreen}{%
\textbf{mort}%
}%
. %
\textcolor{mygreen}{%
\textbf{Il est retiré du jeu pour le reste de la Quête}%
}%
. Le matériel laissé sur place par %
\textcolor{mygreen}{%
\textbf{le Héros décédé peut être récupéré par un allié présent dans la pièce}%
}%
. Si seul un Monstre s'y trouve, il récupère le matériel qui est retiré de la partie.
%
\item[{-} ]%
%
 Quand les points de vie d'un Héros sont réduits à %
\textcolor{mygreen}{%
\textbf{0}%
}%
, il peut échapper à la mort de deux façons :
%
\end{description}%
\begin{itemize}%
\item%
%
 %
\textcolor{mygreen}{%
\textbf{Il peut immédiatement boire une Potion en sa possession et faire remonter ses points de vie au{-}dessus de zéro}%
}%
.
%
\item%
%
 S'il n'a pas encore %
\textcolor{mygreen}{%
\textbf{accompli l'action}%
}%
\textit{ }%
 de son tour, il peut lancer un Sort de guérison en sa possession, sur lui{-}même.
%
\end{itemize}%
\begin{description}%
\item[{-} ]%
%
 %
\textcolor{mygreen}{%
\textbf{Les alliés ne peuvent pas sauver un Héros lors du tour de celui{-}ci}%
}%
.
%
\end{description}%
\begin{itemize}%
\item%
%
 %
\textcolor{mygreen}{%
\textbf{Lancer un Sort de la Terreur}%
}%

%
\end{itemize}%
\begin{description}%
\item[{-} ]%
%
 Zargon peut décider de %
\textcolor{mygreen}{%
\textbf{lancer un Sort de la Terreur}%
}%
\textit{ }%
 plutôt que d'attaquer. Il doit donner ses Sorts à des Monstres spécifiques (cf notes de Quête). Un Monstre peut seulement lancer un Sort sur un Héros %
\textcolor{mygreen}{%
\textbf{qu'il peut voir}%
}%
.
%
\item[{-} ]%
%
 Chaque Sort ne peut être lancé %
\textcolor{mygreen}{%
\textbf{qu'une seule fois}%
}%
\textit{ }%
 par Quête.
%
\item[{-} ]%
%
 Chaque Sort et ses effets sont expliqués en détails sur la carte de Sort correspondante.
%
\end{description}

%
\sectionfont{\color{red}}%
\subsectionfont{\color{red}}%
\subsubsectionfont{\color{red}}%
\section{ Fin de partie
}%
\label{sec:Findepartie}%
\textcolor{red}{\rule{18cm}{0.07cm}}\break%
\begin{description}%
\item[{-} ]%
%
 %
\textcolor{red}{%
\textbf{Une quête est réussie si un héros accomplit la mission et qu'il est revenu à l'escalier.}%
}%

%
\item[{-} ]%
%
 %
\textcolor{red}{%
\textbf{Une quête est échouée si les 4 héros ne survivent pas à la quête.}%
}%

%
\item[{-} ]%
%
 %
\textcolor{red}{%
\textbf{Récupérez la récompense de fin de quête en cas de réussite.}%
}%

%
\item[{-} ]%
%
 %
\textcolor{red}{%
\textbf{Encerclez le numéro de quête correspondant sur votre feuille de personnage.}%
}%
\end{description}

%
\end{document}
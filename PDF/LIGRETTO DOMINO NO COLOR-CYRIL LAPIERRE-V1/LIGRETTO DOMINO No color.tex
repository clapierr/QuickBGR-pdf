\documentclass{article}%
\usepackage[T1]{fontenc}%
\usepackage[utf8]{inputenc}%
\usepackage{lmodern}%
\usepackage{textcomp}%
\usepackage{lastpage}%
\usepackage{geometry}%
\geometry{margin=0.7in}%
\usepackage{fancyhdr}%
%
\usepackage{sectsty}%
\title{LIGRETTO DOMINO No color}%
\author{Cyril Lapierre}%
\date{\today}%
\fancypagestyle{header}{%
\renewcommand{\headrulewidth}{1pt}%
\renewcommand{\footrulewidth}{1pt}%
\fancyhead{%
}%
\fancyfoot{%
}%
\fancyhead[L]{%
Page \thepage%
}%
\fancyhead[C]{%
QuickBGR%
}%
\fancyhead[R]{%
\today%
}%
\fancyfoot[C]{%
QuickBGR%
}%
}%
%
\begin{document}%
\normalsize%
\maketitle\thispagestyle{header}%
\pagestyle{header}%
\section{ Intro
}%
\label{sec:Intro}%
\rule{18cm}{0.07cm}\break%
Posez le plus rapidement vos tuiles domino prés d'un pion Ligretto, tout en respectant les couleurs du plateau. Aidez{-}vous de tuiles joker, accumulez les points lors des différentes manches et tentez de remporter la victoire.


%
\section{ Mise en place
}%
\label{sec:Miseenplace}%
\rule{18cm}{0.07cm}\break

%
\subsection{ Plateau central
}%
\label{subsec:Plateaucentral}%
Assemblez le plateau de jeu selon le %
\textbf{nombre}%
\textit{ }%
 de joueurs et posez{-}le au %
\textbf{centre}%
\textit{ }%
 de la table


%
\subsection{ Chaque joueur reçoit
}%
\label{subsec:Chaquejoueurreoit}%
\begin{description}%
\item[{-} ]%
%
 15 tuiles Domino avec le même %
\textbf{rebord}%
\textit{ }%
 (simple, double, discontinu ou
%
\end{description}%
pointillé). Faites une pile face cachée avec vos 15 tuiles mélangées.
%
\begin{description}%
\item[{-} ]%
%
 2 tuiles carrées %
\textbf{\textit{Joker}}%
\textit{ }%
 dont le %
\textbf{rebord}%
\textit{ }%
 correspond à celui de vos tuiles Domino
%
\item[{-} ]%
%
 1 mini Plateau joueur rouge
%
\item[{-} ]%
%
 1 pion Ligretto
%
\end{description}

%
\section{ Tour de jeu
}%
\label{sec:Tourdejeu}%
\rule{18cm}{0.07cm}\break

%
\subsection{ POSER UN PION LIGRETTO
}%
\label{subsec:POSERUNPIONLIGRETTO}%
Avant chaque manche, les joueurs répartissent ensemble les 4 pions Ligretto sur n’importe quelle case du plateau de jeu.


%
\subsection{ JOUER UNE TUILE DOMINO
}%
\label{subsec:JOUERUNETUILEDOMINO}%
Tous les joueurs jouent %
\textbf{simultanément}%
.
%
Chaque joueur doit poser une tuile sur une case %
\textbf{vide autour}%
\textit{ }%
 d'un pion Ligretto en remplissant les conditions suivantes :
%
\begin{enumerate}%
\item%
%
 La tuile Domino posée doit être %
\textbf{identique}%
\textit{ }%
 aux cases qu'elle recouvre.
%
\item%
%
 Elle ne peut pas %
\textbf{recouvrir}%
\textit{ }%
 une tuile Domino déjà posée.
%
\item%
%
 Elle doit être connectée au pion Ligretto par un %
\textbf{côté}%
.
%
\end{enumerate}%
\begin{description}%
\item[{-} ]%
%
 Une fois votre tuile Domino posée, %
\textbf{déplacez le pion Ligretto}%
\textit{ }%
 sur la couleur %
\textbf{la plus éloignée de son emplacement actuel.
}%
\item[{-} ]%
%
 S'il n'existe aucune possibilité de pose à partir de la nouvelle position d'un pion Ligretto ou si vous enfermez un pion %
\textit{Ligretto}%
\textit{ }%
 déjà posé, vous devez poser votre prochaine tuile Ligretto sur %
\textbf{n’importe quel emplacement}%
\textit{ }%
 du plateau de jeu qui convient pour les deux couleurs. Posez ensuite %
\textbf{immédiatement}%
\textit{ }%
 le pion Ligretto bloqué sur une des deux cases.
%
\item[{-} ]%
%
 Si vous ne trouvez pas d'emplacement pour poser votre tuile Domino, %
\textbf{défaussez{-}la}%
\textit{ }%
 et prenez{-}en une nouvelle.
%
\item[{-} ]%
%
 Si vous désirez poser votre tuile Domino sur un emplacement situé %
\textbf{à une case}%
\textit{ }%
 de distance d'un pion Ligretto, vous pouvez recouvrir cette case avec une de vos %
\textbf{2 tuiles Joker}%
. Suite à quoi, seulement vous, pouvez poser une tuile à côté de ce dernier.
%
\item[{-} ]%
%
 Si vous n'avez %
\textbf{plus de tuiles Domino}%
\textit{ }%
 dans votre pile, formez{-}en une nouvelle avec les tuiles de votre %
\textbf{défausse}%
.
%
\end{description}

%
\subsection{ FIN DE MANCHE
}%
\label{subsec:FINDEMANCHE}%
\begin{description}%
\item[{-} ]%
%
 Dès qu’un joueur a posé toutes ses tuiles Domino sur le plateau de jeu, il crie %
\textbf{\textit{Ligretto Stop}}%
\textit{ }%
 et la manche prend fin.
%
\item[{-} ]%
%
 Vérifiez ensemble si toutes les tuiles ont été %
\textbf{correctement}%
\textit{ }%
 placées. Si un joueur a commis une erreur lors de la pose, il ne participe pas au décompte de points de la manche.
%
\item[{-} ]%
%
 Le joueur qui a posé la %
\textbf{totalité}%
\textit{ }%
 de ses tuiles Domino reçoit un jeton %
\textbf{2 P.V}%
.
%
\item[{-} ]%
%
 Parmi les joueurs restants, celui qui a le %
\textbf{moins}%
\textit{ }%
 de tuiles Domino en sa possession prend un jeton %
\textbf{1 P.V}%
.
%
\item[{-} ]%
%
 Les joueurs qui n'ont pas obtenus de points posent %
\textbf{2 de leurs tuiles Domino}%
\textit{ }%
 sur leur %
\textit{Plateau joueur}%
\textit{ }%
 rouge. Ils partiront avec moins de tuiles lors de la prochaine manche
%
\end{description}

%
\section{ Fin de partie
}%
\label{sec:Findepartie}%
\rule{18cm}{0.07cm}\break%
\textbf{Avant le début de la partie, les joueurs décident du nombre de manches qu’ils souhaitent jouer. Le joueur qui a obtenu le plus de points à la fin du nombre de manches décidé remporte la partie.}

%
\end{document}
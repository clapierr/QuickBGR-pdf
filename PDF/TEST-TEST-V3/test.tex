\documentclass{article}%
\usepackage[T1]{fontenc}%
\usepackage[utf8]{inputenc}%
\usepackage{lmodern}%
\usepackage{textcomp}%
\usepackage{lastpage}%
\usepackage{geometry}%
\geometry{margin=0.7in}%
\usepackage{xcolor}%
\usepackage{fancyhdr}%
%
\usepackage{sectsty}%
\usepackage{xcolor}%
\definecolor{mygreen}{RGB}{50,150,50}%
\title{test}%
\author{test}%
\date{\today}%
\fancypagestyle{header}{%
\renewcommand{\headrulewidth}{1pt}%
\renewcommand{\footrulewidth}{1pt}%
\fancyhead{%
}%
\fancyfoot{%
}%
\fancyhead[L]{%
Page \thepage%
}%
\fancyhead[C]{%
QuickBGR%
}%
\fancyhead[R]{%
\today%
}%
\fancyfoot[C]{%
QuickBGR%
}%
}%
%
\begin{document}%
\normalsize%
\maketitle\thispagestyle{header}%
\pagestyle{header}%
\sectionfont{\color{blue}}%
\subsectionfont{\color{blue}}%
\subsubsectionfont{\color{blue}}%
\section{ Intro
}%
\label{sec:Intro}%
\textcolor{blue}{\rule{18cm}{0.07cm}}\break%
Échangez les cartes de votre tableau avec celles de la pioche ou de la défausse et tentez d'avoir la plus petite valeur. Essayez de constituer des colonnes de 3 cartes identiques pour les éliminer et augmenter ainsi votre chance d'avoir le plus petit score à la fin de la manche.


%
\sectionfont{\color{mygreen}}%
\subsectionfont{\color{mygreen}}%
\subsubsectionfont{\color{mygreen}}%
\section{ Mise en place
}%
\label{sec:Miseenplace}%
\textcolor{mygreen}{\rule{18cm}{0.07cm}}\break%
\begin{description}%
\item[{-} ]%
%
 Chaque joueur reçoit %
\textcolor{mygreen}{%
\textbf{12 cartes}%
}%
\textit{ }%
 face %
\textcolor{mygreen}{%
\textbf{cachée}%
}%
.
%
\item[{-} ]%
%
 Avec vos 12 cartes toujours face cachée, constituez un tableau de 4 colonnes de 3 cartes.
%
\item[{-} ]%
%
 Retournez %
\textcolor{mygreen}{%
\textbf{face visible, 2 des 12 cartes}%
}%
\textit{ }%
 de votre tableau.
%
\item[{-} ]%
%
 Formez une %
\textcolor{mygreen}{%
\textbf{pioche}%
}%
\textit{ }%
 avec les cartes restantes.
%
\item[{-} ]%
%
 Révélez la première carte de la pioche, elle constitue le début de la %
\textcolor{mygreen}{%
\textbf{pile de défausse}%
}%
.
%
\item[{-} ]%
%
 Le joueur, dont la somme des 2 cartes visibles est la plus élevée, commence la partie.
%
\end{description}

%
\sectionfont{\color{red}}%
\subsectionfont{\color{red}}%
\subsubsectionfont{\color{red}}%
\section{ Tour de jeu
}%
\label{sec:Tourdejeu}%
\textcolor{red}{\rule{18cm}{0.07cm}}\break%
LE JEU SE DÉROULE EN PLUSIEURS MANCHES SUCCESSIVES. À CHAQUE TOUR UN JOUEUR DOIT CHOISIR UNE CARTE ENTRE LA DÉFAUSSE OU LA PIOCHE.


%
\subsection{ PRENDRE LA PREMIÈRE CARTE RETOURNÉE DE LA DÉFAUSSE
}%
\label{subsec:PRENDRELAPREMIRECARTERETOURNEDELADFAUSSE}%
\begin{description}%
\item[{-} ]%
%
 Vous devez %
\textcolor{red}{%
\textbf{immédiatement}%
}%
\textit{ }%
 échanger cette carte avec l’une de vos 12 cartes (visible ou cachée) :
%
\end{description}%
\begin{itemize}%
\item%
%
 La nouvelle carte est posée face visible.
%
\item%
%
 L'ancienne carte est déposée sur le dessus de la pile de défausse, face visible.
%
\end{itemize}%
\begin{description}%
\item[{-} ]%
%
 Vous ne pouvez pas regarder vos cartes cachées avant de faire un échange.
%
\end{description}

%
\subsection{ PRENDRE LA PREMIÈRE CARTE FACE CACHÉE DE LA PIOCHE
}%
\label{subsec:PRENDRELAPREMIRECARTEFACECACHEDELAPIOCHE}%
\begin{description}%
\item[{-} ]%
%
Une fois la carte piochée, vous pouvez %
\textcolor{red}{%
\textbf{au choix}%
}%
\textit{ }%
 :
%
\end{description}%
\textit{ }%
\textit{L'échanger}%
\textit{ contre l'une de vos 12 cartes (visible ou cachée).
}%
\textit{ La déposer sur le dessus de la pile de }%
\textit{défausse}%
\textit{ car elle ne vous intéresse pas. Dans ce cas, vous devez retourner face visible une de vos cartes cachées.
}

%
\subsection{ RÈGLE SPÉCIALE
}%
\label{subsec:RGLESPCIALE}%
\begin{description}%
\item[{-} ]%
%
 Si suite à un échange ou après avoir révélé une carte face cachée, vous possédez %
\textcolor{red}{%
\textbf{3 cartes identique}%
}%
s dans une colonne, %
\textcolor{red}{%
\textbf{retirez}%
}%
\textit{ }%
 ces 3 cartes de votre tableau et placez{-}les sur la défausse.
%
\end{description}

%
\sectionfont{\color{cyan}}%
\subsectionfont{\color{cyan}}%
\subsubsectionfont{\color{cyan}}%
\section{ Fin de la manche
}%
\label{sec:Findelamanche}%
\textcolor{cyan}{\rule{18cm}{0.07cm}}\break%
\begin{description}%
\item[{-} ]%
%
Le premier joueur à révéler toutes ses cartes (%
\textit{toutes les cartes sont face visible}%
), met fin à la manche.
%
\item[{-} ]%
%
 Finissez le tour en cours.
%
\item[{-} ]%
%
 Les joueurs ayant encore des %
\textcolor{cyan}{%
\textbf{cartes face cachée}%
}%
\textit{ }%
 les retournent face visible. Appliquez la %
\textcolor{cyan}{\textbf{\textit{Règle spéciale}}}%
\textit{ }%
 des 3 cartes identiques si le cas se présente.
%
\item[{-} ]%
%
 Chaque joueur comptabilise son %
\textcolor{cyan}{%
\textbf{total de points}%
}%
\textit{ }%
 en ajoutant toutes les valeurs des cartes visibles dans son tableau:
%
\end{description}%
\textit{ Si le total de points est }%
\textit{positif}%
\textit{, vous devez le }%
\textit{rajouter}%
\textit{ à votre total de points des manches }%
précédentes%
\textit{.
}%
\textit{ Si le total est }%
\textit{négatif}%
\textit{, vous devez le soustraire à votre total de points des manches précédentes.
}%
\begin{description}%
\item[{-} ]%
%
Si le joueur qui a terminé la manche, n’a pas strictement obtenu le plus petit nombre de points de cette manche,alors, la somme de ses points pour cette manche est %
\textcolor{cyan}{%
\textbf{doublée}%
}%
.
%
\end{description}

%
\sectionfont{\color{orange}}%
\subsectionfont{\color{orange}}%
\subsubsectionfont{\color{orange}}%
\section{ Fin de la partie
}%
\label{sec:Findelapartie}%
\textcolor{orange}{\rule{18cm}{0.07cm}}\break%
\begin{description}%
\item[{-} ]%
%
 %
\textcolor{orange}{%
\textbf{Le jeu se termine dès qu’un joueur atteint 100 points ou plus.}%
}%

%
\item[{-} ]%
%
 %
\textcolor{orange}{%
\textbf{Le joueur avec le plus petit score gagne la partie.}%
}%
\end{description}

%
\end{document}